%theory

\part{Golden Dual Fullerenes}
\label{sec:goldendualfullerenes}


\chapter{Theory}
\label{sec:theorygold}

\section{Basics of Quantum Chemistry}
\label{sec:basicsofQC}

In this part of the thesis knowledge of standard methods used in quantum
chemistry is required. The following chapters introduce the fundamentals of
quantum mechanics and their application to chemistry related questions. Most of
it is based on standard literature by
Jensen\autocite{JensenIntroductionComputationalChemistry2006}, Szabo and
Ostlund\autocite{SzaboModernQuantumChemistry1996} and Holthausen and
Koch\autocite{KochChemistGuideDensity2001}.

\subsection{The Schr\"odinger Equation}
\label{sec:schrodingerequation}

The beginning of the 20th centuary was a very important time for modern
theoretical sciences. Discoveries like Planck's energy quantisation based on
black body radiation or the discovery of the wave particle dualism by de
Broglie lead to a complete reformulation of the physical laws governing the
smallest of particles. Erwin Schr\"odinger established a framework based on
Hamiltonian mechanics that set the wave function $\Psi$ at the center of
attention. It contains all information about the system and its evolution in
time. For ground-state calculations it is usually sufficient to look at
solutions to the time-independent Schr\"odinger equation.

\begin{equation}
	\mathbf{H}\Psi=E\Psi\label{eqn:SchrodingerEquation}
\end{equation}

In this eigenvalue equation the Hamilton operator $\mathbf{H}$ acts on the wave
function which results in a solution for the total energyi $E$ of the system. 

The wave function is a function of all spatial $\mathbf{r}$ and spin
coordinates $\omega$ of all the particles in the system. The combination of
these coordinates will be denoted $\mathbf{x}$ on the following pages.  

\begin{align}
	\mathbf{x}=\{\mathbf{r},\omega\}
\end{align}

The square of the wave function is usually interpreted as a probability density
which is used to normalise it such that the probability of finding an electron
anywhere in space is one.

\begin{align}
	\int |\Psi(\mathbf{x})|^2\dif \mathbf{x}_1\dif \mathbf{x}_2\ldots\dif \mathbf{x}_N=\braket{\Psi(\mathbf{x})|\Psi(\mathbf{x})}=1
\end{align}

The information contained in the wave function can be accessed by an operator
$\mathbf{O}$ acting on the wave function and forming an eigenvalue equation. In
equation~\eqref{eqn:SchrodingerEquation} this operator is the Hamilton
operator, but it could be any operator connected to a physical observable.
Generally, the eigenvalue of an arbitrary operator $\braket{\mathbf{O}}$ is
defined as

\begin{equation}
	\braket{\mathbf{O}}=\frac{\braket{\Psi(\mathbf{x})|\mathbf{O}|\Psi(\mathbf{x})}}{\braket{\Psi(\mathbf{x})|\Psi(\mathbf{x})}},
\end{equation}

or for normalised wave functions

\begin{align}
	\braket{\mathbf{O}}=\braket{\Psi(\mathbf{x})|\mathbf{O}|\Psi(\mathbf{x})}.
\end{align}

For a system of $N$ electrons and $M$ nuclei the non-relativistic Hamilton
operator in atomic units has the form

\begin{align}
\begin{aligned}
    \mathbf{H}&=\mathbf{T}_\text{e} + \mathbf{V}_\text{ee} + \mathbf{T}_\text{n} + \mathbf{V}_\text{nn} + \mathbf{V}_\text{en} \\
    &=-\sum_{i=1}^N\frac{1}{2}\nabla_i^2
    + \sum_{i=1}^N\sum_{j>i}^NV_{ij}
    - \sum_{A=1}^M\frac{1}{2}\nabla_A^2
    + \sum_{A=1}^M\sum_{B>A}^MV_{AB}
    + \sum_{i=1}^N\sum_{A=1}^MV_{iA}.\label{eqn:hamiltonoperator}
\end{aligned}
\end{align}

The operator contains the kinetic energy of the electrons
$\mathbf{T}_\text{e}$, the electron repulsion $\mathbf{V}_\text{ee}$, the
kinetic energy of the nuclei $\mathbf{T}_\text{n}$, the nucleon repulsion
$\mathbf{V}_\text{nn}$ and the electron-nucleus attraction
$\mathbf{V}_\text{en}$.

\subsection{The Born-Oppenheimer Approximation}
\label{sec:bornoppenheimerapproximation}

For quantum chemical applications the coupling of the movement of the electrons
and nuclei is usually neglected. This is possible because the atomic mass $m_A$
is so much greater than the electronic mass $m_i$, resulting in much lower
velocities for nuclei compared to electrons. Therefore, the electrons can be
considered to be moving in a static field of nuclei, meaning the nuclear kinetic
term $\mathbf{T}_\text{n}$ can be neglected and the nuclear repulsion term
$\mathbf{V}_\text{nn}$ becomes a constant. The resulting electronic Hamilton
operator $\mathbf{H}_\text{e}$ describes electrons moving in a field of
positive point charges.

\begin{align}
    \mathbf{H}_\text{e}=\mathbf{T}_\text{e} + \mathbf{V}_\text{ee} + \mathbf{V}_\text{en}
\end{align}

Solving the Schr\"dinger equation for this operator yields the electronic wave
function $\Psi_\text{e}$. It depends explicitly on the electronic coordinates,
but only parametrically on the nuclear coordinates, therefore spanning a
potential energy surface upon which the nuclei move. To get the total energy of
the system the constant repulsion between the nuclei has to be added to the
electronic operator. 

\begin{align}
    \mathbf{H}=\mathbf{H}_\text{e}+V_\text{nn}\label{eqn:hamiltonoperatorfinal}
\end{align}

\section{The Hartree-Fock Approximation}
\label{sec:hartreefockapproximation}

An exact solution to the Schr\"odinger equation in the boundary conditions of
the Born-Oppenheimer approximation for systems containing more than one
electron is not possible. The solution has to be approximated by appropriate
methods; one possible solution are the Hartree-Fock equations. This set of
equations determines the energy of a wave function expressed as a Slater
determinant $\Phi_\text{SD}$.

\begin{equation}
     \Phi_{SD}=\frac{1}{\sqrt{N!}}
     \begin{vmatrix}
         \phi_1(\mathbf{x}_1) & \phi_2(\mathbf{x}_1) & \cdots & \phi_N(\mathbf{x}_1)\\
         \phi_1(\mathbf{x}_2) & \phi_2 (\mathbf{x}_2) & \cdots & \phi_N(\mathbf{x}_2)\\
         \vdots & \vdots & \ddots & \vdots\\
         \phi_1(\mathbf{x}_N) & \phi_2(\mathbf{x}_N) & \cdots & \phi_N(\mathbf{x}_N)
     \end{vmatrix}
     \label{eqn:SlaterDet}
\end{equation}

Here, the $\phi_i$ denote one electron spin orbitals and
$\mathbf{x}_i=\{\mathbf{r}_i,\omega_i\}$ are spatial ($\mathbf{r}_i$) and spin
($\omega_i$) coordinates of the electrons. A Slater determinant obeys the Pauli
exclusion principle, which requires the electronic wave function to be
anti-symmetric with respect to interchanging the coordinates of two electrons.

In the Hartree-Fock approximation the energy of a single Slater determinant is
used as an approximation for the total energy of the system. As explained
previously, the energy of a wave function can be determined by the action of
the Hamilton operator on the wave function. The Hamilton operator in the
Born-Oppenheimer approximation from equation~\eqref{eqn:hamiltonoperatorfinal}
can be rewritten in terms of one-electron operators $\mathbf{h}_i$.

\begin{align}
\begin{aligned}
    \mathbf{H}&=\sum_i^N\left(\mathbf{h}_i + \sum_{j>i}^N\frac{1}{|\mathbf{r}_i-\mathbf{r}_j|}\right) + V_\text{nn} \\
    \mathbf{h}_i&=-\frac{1}{2}\nabla_i^2-\sum_A^M\frac{Z_A}{|\mathbf{r}_A-\mathbf{r}_i|}\label{eqn:hamiltonoperatorhartreefock}
\end{aligned}
\end{align}

Here, $Z_A$ denotes the charge of nucleus $A$. $\mathbf{h}_i$ depends only on
the kinetic energy of electron $i$ and its potential energy in the field of all
$M$ nuclei. When $\mathbf{h}_i$ acts on a Slater determinant the result is the
respective matrix element $h_i$. Only parts of the Slater determinant without
permutation of electron coordinates can give a non-zero contribution to the
eigenvalue.

\begin{align}
    \begin{aligned}
        h_1&=\braket{\phi_1(\mathbf{x}_1)\phi_2(\mathbf{x}_2)\ldots\phi_N(\mathbf{x}_N)|\mathbf{h}_1|\phi_1(\mathbf{x}_1)\phi_2(\mathbf{x}_2)\ldots\phi_N(\mathbf{x}_N)}\\
        &=\braket{\phi_1(\mathbf{x}_1)|\mathbf{h}_1|\phi_1(\mathbf{x}_1)}\braket{\phi_2(\mathbf{x}_2)|\phi_2(\mathbf{x}_2)}\ldots\braket{\phi_N(\mathbf{x}_N)|\phi_N(\mathbf{x}_N)} \\
        &=\braket{\phi_1(\mathbf{x}_1)|\mathbf{h}_1|\phi_1(\mathbf{x}_1)}
    \end{aligned}
\end{align}

The remainder of the Hamilton operator from
equation~\eqref{eqn:hamiltonoperatorhartreefock} is dependant on two electron
coordinates. It is convenient to define a two-electron operator $\mathbf{g}_{ij}$
with the matrix elements $g_{ij}=\frac{1}{|\mathbf{r}_i-\mathbf{r}_j|}$. Its
action on the part of the Slater determinant with no permutation of electron
coordinates results in the Coulomb integral and the corresponding matrix
element $J_{ij}$, which can be interpreted as the classical Coulomb repulsion.

\begin{align}
    \begin{aligned}
        J_{12}&=\braket{\phi_1(\mathbf{x}_1)\phi_2(\mathbf{x}_2)\ldots\phi_N(\mathbf{x}_N)|\mathbf{g}_{12}|\phi_1(\mathbf{x}_1)\phi_2(\mathbf{x}_2)\ldots\phi_N(\mathbf{x}_N)}\\
        &=\braket{\phi_1(\mathbf{x}_1)\phi_2(\mathbf{x}_2)|\mathbf{g}_{12}|\phi_1(\mathbf{x}_1)\phi_2(\mathbf{x}_2)}\ldots\braket{\phi_N(\mathbf{x}_N)|\phi_N(\mathbf{x}_N)} \\
        &=\braket{\phi_1(\mathbf{x}_1)\phi_2(\mathbf{x}_2)|\mathbf{g}_{12}|\phi_1(\mathbf{x}_1)\phi_2(\mathbf{x}_2)}
    \end{aligned}
\end{align}

As $\mathbf{g}_{ij}$ depends on the coordinates of two electrons it also yields
non-zero matrix elements for parts of the Slater determinant, where two
electron coordinates have been swapped.

\begin{align}
    \begin{aligned}
        K_{12}&=\braket{\phi_1(\mathbf{x}_1)\phi_2(\mathbf{x}_2)\ldots\phi_N(\mathbf{x}_N)|\mathbf{g}_{12}|\phi_2(\mathbf{x}_1)\phi_1(\mathbf{x}_2)\ldots\phi_N(\mathbf{x}_N)}\\
        &=\braket{\phi_1(\mathbf{x}_1)\phi_2(\mathbf{x}_2)|\mathbf{g}_{12}|\phi_2(\mathbf{x}_1)\phi_1(\mathbf{x}_2)}\ldots\braket{\phi_N(\mathbf{x}_N)|\phi_N(\mathbf{x}_N)} \\
        &=\braket{\phi_1(\mathbf{x}_1)\phi_2(\mathbf{x}_2)|\mathbf{g}_{12}|\phi_2(\mathbf{x}_1)\phi_1(\mathbf{x}_2)}
    \end{aligned}
\end{align}

$K_{ij}$ is called the exchange integral and has no classical interpretation.
As swapping coordinates in the Slater determinant changes its sign, the result
of the exchange integral has a negative sign. The total energy of the system is
now given by the sum over all integrals described above.

\begin{align}
    E=\sum_{i=1}^Nh_i + \frac{1}{2}\sum_{i=1}^N\sum_{j=1}^N(J_{ij}-K_{ij})+V_\text{nn}\label{eqn:HFenergy}
\end{align}

Defining operators $\mathbf{J}_i$ and $\mathbf{K}_i$ for Coulomb and exchange
integral equation~\eqref{eqn:HFenergy} becomes

\begin{align}
    E=\sum_{i=1}^N\braket{\phi_i|\mathbf{h}_i|\phi_i} + \frac{1}{2}\sum_{i=1}^N\sum_{j=1}^N \left( \braket{\phi_i|\mathbf{J}_j|\phi_i} - \braket{\phi_i|\mathbf{K}_j|\phi_i} \right) + V_\text{nn}\\
    \begin{aligned}
		\mathbf{J}_i\ket{\phi_j(\mathbf{x}_j)}&=\braket{\phi_i(\mathbf{x}_i)|\mathbf{g}_{ij}|\phi_i(\mathbf{x}_i)}\ket{\phi_j(\mathbf{x}_j)}\\
		\mathbf{K}_i\ket{\phi_j(\mathbf{x}_j)}&=\braket{\phi_i(\mathbf{x}_i)|\mathbf{g}_{ij}|\phi_j(\mathbf{x}_i)}\ket{\phi_i(\mathbf{x}_j)}.
    \end{aligned}\label{eqn:HFenergyoperator}
\end{align}

For the purpose of quantum chemical calculations the energy of an arbitrary
Slater determinant is usually not useful. More interesting, however, is to find
the Slater determinant that minimises the energy under the boundary condition
of keeping the orthonormality condition between spin orbitals. In other words,
we try to find the derivative of equation~\eqref{eqn:HFenergyoperator}.
Minimising the energy under external boundary conditions can be solved using
Lagrange multipliers. With their help it is possible to define the Fock
operator $\mathbf{f}_i$, which is an effective one-electron operator.

\begin{align}
    \mathbf{f}_i=\mathbf{h}_i + \sum_j^N\left( \mathbf{J}_j - \mathbf{K}_j \right)
\end{align}

The action of the Fock operator on an element of the Slater determinant yields
the Hartree-Fock equations.

\begin{align}
    \mathbf{f}_i\ket{\phi_i}=\sum_j^N\lambda_{ij}\ket{\phi_j}\label{eqn:HFequations}
\end{align}

$\lambda_{ij}$ are Lagrange multipliers remaining from the constrained
minimisation. They can be re-written in matrix form and subsequently
diagonalised by a unitary transformation. This yields the canonical
Hartree-Fock equations.

\begin{align}
    \mathbf{f}_i\ket{\phi'_i}=\sum_j^N\varepsilon_i\ket{\phi'_j}\label{eqn:HFequationscanonical}
\end{align}

$\varepsilon_i$ are orbital energies of the electrons. According to Koopman's
theorem they can be interpreted as ionisation energies for occupied and
electron affinities for unoccupied states. The Fock operator depends on all
occupied states, making it a pseudo eigenvalue equation. Hence, solutions have
to be found iteratively starting from and arbitrary set of orbitals. After a
set of convergence criteria has been met, the effective potential is said to
remain unchanged, creating a \ac{SCF} solution.  \\\newline Solving the
canonical Hartree-Fock equations for larger systems is possible, but costly.
Usually they are solved using a basis set expansion to approximate the unknown
molecular orbitals. The basis functions are usually chosen to agree with the
underlying physics of the system. For example, periodic plane waves are
usually used when periodic boundary conditions are required. For calculations
in the gas phase the basis functions are usually exponential functions
centered at the nuclei. In this case the approximation is called \ac{LCAO}.
Technically, a basis set expansion is not an approximation, but as one is
limited to a finite amount of basis functions $P$ the expansion does not give
an exact expression for a molecular orbital $\phi_i$. For a set of $\alpha$
basis functions $\chi_\alpha$ the expansion can be expressed as follows.

\begin{align}
    \phi_i=\sum_\alpha^Pc_{\alpha i}\chi_\alpha
\end{align}

This leads to the Hartree-Fock equations expressed in the basis set
approximation.

\begin{align}
    \begin{aligned}
        \mathbf{f}_i\sum_\alpha^Pc_{\alpha i}\chi_\alpha&=\varepsilon_i\sum_\alpha^Pc_{\alpha i}\chi_\alpha \\
        \sum_\alpha^Pc_{\alpha i}\underbrace{\braket{\chi_\alpha|\mathbf{f}_i|\chi_\beta}}_{F_{\alpha\beta}}&=\varepsilon_i\sum_\alpha^Pc_{\alpha i}\underbrace{\braket{\chi_\alpha|\chi_\beta}}_{S_{\alpha\beta}}
    \end{aligned}
\end{align}

These are the Roothaan-Hall equations which are usually written in matrix form.

\begin{align}
    \mathbf{FC}=\mathbf{SC\varepsilon}
\end{align}

$\mathbf{F}$ is the Fock matrix, $\mathbf{S}$ is the overlap matrix and
$\mathbf{C}$ contains the orbital coefficients. These equations have to be
solved iteratively and to reduce computational cost the first step usually
involves calculating a density matrix $\mathbf{D}$.

\begin{align}
    \mathbf{D}=\sum_{i}^\text{occ. MO} c_{\mu i}c_{\nu i}
\end{align}

$\mathbf{D}$ can be used to generate a Fock matrix, which will be diagonalised
yielding a new set of orbital coefficients. These will be used to generate a
new generation of the density matrix. This procedure will be repeated until
the coefficients of the new generation are equal (up to a certain precision)
to the ones of the parent generation. This marks the end of the \ac{SCF} cycle.

\section{Density Functional Theory}
\label{sec:dft}

The Hartree-Fock method belongs to the class of mean field approximations,
which means the electrons don't interact directly with each other, but each
electron is moving in a field created by all other electrons. The Hartree-Fock
energy $E_\text{HF}$ is therefore never exact even in the infinite basis set
limit. The difference to the exact energy $E_0$ was first named correlation
energy $E_\text{corr}$ by L\"owdin\autocite{lowdin_correlation_1958}.

\begin{align}
    E_\text{corr} = E_0 - E_\text{HF}
\end{align}

Even though the correlation energy only attributes for about \SI{1}{\percent}
of the total energy it is an important contribution in molecular systems. A
multitude of methods has been developed to treat the correlation energy more
accurately. They can generally be subdivided into post-Hartree-Fock methods and
\ac{DFT}. While post-Hartree-Fock methods rely on a Hartree-Fock wave function
as a starting point, \ac{DFT} is in principle a wave function free method. It
establishes a connection between the energy of the system and the electron
density $\rho$ instead of the wave function. The electron density is observable
and a positive quantity, which makes \ac{DFT} easier to grasp than wave
function based methods. However, the electron density is related to the wave
function via its square. 

\begin{align}
    \rho(\mathbf{r}_1)=N\int\dots\int|\Psi(\mathbf{x}_1,\mathbf{x}_2,\dots,\mathbf{x}_N)|^2\;\dif\omega_1\dif\mathbf{x}_2\dots\dif\mathbf{x}_N\label{eqn:electrondensity}
\end{align}

Equation~\eqref{eqn:electrondensity} describes the probability density of
finding one of the $N$ electrons in the volume $\dif \mathbf{r}_1$ with
arbitrary spin. All other $(N-1)$ electrons occupy any volume elements, but
have opposite spin. As electrons are indistinguishable the probability to find
any of the $N$ electrons in the volume $\dif\mathbf{r}_1$ is equal to $N$ times
the probability of finding a specific electron in that volume.  A justification
for using the electron density instead of the wave function was found by
Hohenberg and Kohn in 1964.

\subsection{Hohenberg-Kohn Theorems}
\label{sec:HKtheorems}

The first Hohenberg-Kohn theorem implies that the electron density defines a
unique external potential that contains all information about the system. If
two external potentials are different they cannot lead to the same ground state
electron density. From this, the total energy of a system can be expressed as follows.

\begin{align}
    E_0\left[ \rho_0(\mathbf{r}) \right] = T\left[ \rho_0(\mathbf{r}) \right] + E_\text{ee} \left[ \rho_0(\mathbf{r}) \right] + E_\text{Ne} \left[ \rho_0(\mathbf{r}) \right]\label{eqn:energyDFT}
\end{align}

The aforementioned external potential corresponds to $E_\text{Ne} \left[
	\rho_0(\mathbf{r}) \right]$, which is the only system-dependent term of
equation~\eqref{eqn:energyDFT}. The system-independent terms for the kinetic
energy $T\left[ \rho_0(\mathbf{r}) \right]$ and the electron-electron
interaction $E_\text{ee} \left[ \rho_0(  \mathbf{r}) \right]$ can be combined
to the Hohenberg-Kohn functional $F_\text{HK}\left[ \rho_0(
\mathbf{r})\right]$.

\begin{align}
    E_0\left[ \rho_0(\mathbf{r}) \right] = \int\dif\mathbf{r}\rho_0(\mathbf{r})V_\text{Ne}+ F_\text{HK}\left[ \rho_0(\mathbf{r})\right]\label{eqn:energyDFTHK}
\end{align}

Would the exact Hohenberg-Kohn functional be known
equation~\eqref{eqn:energyDFTHK} would be an exact solution to the
Schr\"odinger equation. In practice, there are no exact expressions for
$T\left[ \rho_0(\mathbf{r}) \right]$ or $E_\text{ee} \left[ \rho_0(
\mathbf{r}) \right]$ knwon. However, the latter can be expressed in terms of a
classical Coulomb term $J\left[\rho\right]$ and a non-classical energy contribution
$E_\text{nc}\left[\rho\right]$.

\begin{align}
    E_\text{ee} \left[ \rho \right]=J\left[\rho\right]+E_\text{nc}\left[\rho\right]=\frac{1}{2}\iint\frac{\rho(\mathbf{r}_1)\rho(\mathbf{r}_2)}{r_{12}}\dif\mathbf{r}_1\dif\mathbf{r}_2+E_\text{nc}\left[\rho\right]
\end{align}

The second Hohenberg-Kohn theorem warrants that a trial energy density
$\widetilde{\rho}$ always yields an energy greater or equal to the exact ground
state energy.

\begin{align}
    E_0 \leq E\left[\widetilde{\rho}\right]
\end{align}

It is equivalent to the variational theorem\footnote{The variational theorem
states that no trial wave function can result in a smaller energy than the exact
ground state wave function.} in wave function theory. However, it is only valid
for the exact Hohenberg-Kohn functional, which is unknown. A solution to this
problem was given in 1965 by Kohn and Sham.

\subsection{Kohn-Sham Theory}
\label{sec:kohnshamtheory}

One of the biggest flaws in orbital free \ac{DFT} ist the poor description of
the kinetic energy term. Kohn and Sham realised it would be easier to describe
it in terms of a reference system of non-interacting electrons. Their kinetic
energy $T_S$ can be expressed in terms of a Slater determinant consisting of
one-electron orbitals $\phi_i$ also called Kohn-Sham orbitals.

\begin{align}
    T_S=-\frac{1}{2}\sum_i^N\braket{\phi_i|\nabla_i^2|\phi_i}
\end{align}

The electron density resulting from the Kohn-Sham orbitals is required to be
equal to the density of the real system.

\begin{align}
    \rho_S(\mathbf{r}) = \rho_0(\mathbf{r})
\end{align}

For such a system the Coulomb interaction between electrons and nuclei can be
calculated exactly. The only unknown remaining terms are the non-classical
contribution to the electron-electron interaction $E_\text{nc}$ and a
contribution to the kinetic energy because of electron correlation $T_C$. These
terms can be combined to the exchange-correlation term $E_{XC}$.

\begin{align}
    E\left[\rho\right]=T_S\left[\rho\right] + J\left[\rho\right] + E_{XC}\left[\rho\right] + E_\text{Ne}\left[\rho\right] \\
    \begin{aligned}
    =&-\frac{1}{2}\sum_i^N\braket{\phi_i|\nabla_i^2|\phi_i} + \frac{1}{2}\sum_i^N\sum_j^N\iint \dif\mathbf{r}_1\dif\mathbf{r}_2|\phi_i(\mathbf{r}_1)|^2\frac{1}{r_{12}}|\phi_j(\mathbf{r}_2|^2 \\
        &+ E_{XC}\left[\rho\right] - \sum_i^N\int\dif\mathbf{r}_1\sum_A^M|\phi_i(\mathbf{r}_1)|^2
    \end{aligned}\label{eqn:energyKS}
\end{align}

Similar to Hartree-Fock theory, the minimal energy can be calculated using
Lagrange multipliers. The potential terms from equation~\eqref{eqn:energyKS}
can be combined to an effective potential $V_S$, which allows for the
definition of a Kohn-Sham operator $\mathbf{f}_\text{KS}$ analogous to the Fock
operator in Hartree-Fock theory.

\begin{align}
    V_S(\mathbf{r}_1) = \int\dif\mathbf{r}_2\frac{\rho(\mathbf{r}_2i)}{r_{12}} + V_{XC}(\mathbf{r}_1) - \sum_A^M\frac{Z_A}{r_{1A}}\\
    \mathbf{f}_\text{KS} = -\frac{1}{2}\nabla^2+V_S(\mathbf{r}_1)\\
    \mathbf{f}_\text{KS}\phi_i=\varepsilon_i\phi_i
\end{align}

These are the Kohn-Sham equations and they have to be solved in an iterative
procedure, because of the Kohn-Sham operator depending on the occupied
orbitals.  The unknown Kohn-Sham orbitals are usually expanded in terms of
basis functions such that the equations can be expressed in matrix form,
similar to the Roothaan-Hall equations. The Fock matrix is replaced by the
Kohn-Sham matrix $\mathbf{F}_\text{KS}$.

\begin{align}
    \mathbf{F}_\text{KS}\mathbf{C} = \mathbf{SC\varepsilon}
\end{align}

\subsection{Exchange and Correlation Functionals}
\label{sec:exchangecorrelationfunctionals}

The key to solving the Kohn-Sham equations is the exchange-correlation energy
$E_\text{XC}$. Over the years there have been lots of proposals for its form,
the oldest being the \ac{LDA}. It is based on the uniform electron gas for
which analytical functionals for exchange and correlation are known.

\begin{align}
    E_{XC}^\text{LDA}\left[\rho\right]=\int\rho(\mathbf{r})\varepsilon_{XC}\left[\rho(\mathbf{r})\right]\dif\mathbf{r}
\end{align}

The exchange-correlation energy $\varepsilon_{XC}\left[\rho(\mathbf{r})\right]$
is weighted with the probability of finding and electron at this point in
space. After separation of the exchange and correlation parts the exchange
energy can be described by a term developed by Slater.

\begin{align}
	\varepsilon_{XC}\left[\rho(\mathbf{r})\right] = \varepsilon_{X}\left[\rho(\mathbf{r})\right] + \varepsilon_{C}\left[\rho(\mathbf{r})\right]\\
	E_{X}^\text{LDA}\left[\rho\right] = -C_X\int\rho(\mathbf{r})^{\frac{4}{3}}\dif\mathbf{r}
\end{align}

No simple formula for the correlation term is known.

\ac{LDA} describes the inhomogeneous electron density by dividing it up into
small homogeneous volumes. An improvement over \ac{LDA} can be made if the
homogeneous electron density is expanded in a Taylor series. Truncating after the
first term gives the \ac{LDA} approximation, including one more term is called
the \ac{GEA}. Because \ac{GEA} doesn't correctly describe the
exchange-correlation hole function it gives worse results that \ac{LDA}.

A break-through for theoretical chemistry has been achieved with the
introduction of the \ac{GGA}. It uses the \ac{GEA} hole functions and tailors
them to physically meaningful boundary conditions. 

In this work the PBE functional by Perdew, Burke and Ernzerhof\autocite{PerdewGeneralizedGradientApproximation1996,PerdewGeneralizedGradientApproximation1997},
which belongs to the group of \ac{GGA} functionals, was used for most
calculations. They published both, correlation and exchange expressions for
this functional.
%%show functional

\subsection{Dispersion Corrections}
\label{sec:dispersioncorrections}

Long-range dispersive effects are part of the correlation energy and most
\ac{DFT} functionals can only describe these effects to very limited degree.
Grimme \textit{et al.} developed a method that can be used in conjunction with
most density functionals. Based on the functional used it calculates a
dispersive energy contribution $E_\text{disp}$ (and gradient contribution for
optimisations) that can be added to the \ac{DFT} energy $E_\text{DFT}$. The
dispersion energy is always of attractive nature and therefore has a negative
sign by convention.

\begin{align}
	E = E_\text{DFT} + E_\text{disp}
\end{align}

In the third generation D3 dispersion correction, which was used in this work,
the calculation of $E_\text{disp}$ involves solving a two- and three-body term.
The two-body term $E^{(2)}$ is more important and is only a function of the
distance between two nuclei $r_{AB}$.

\begin{align}
	E^{(2)}=\sum\limits_{AB}s_6\frac{C_6^{AB}}{\left(r_{AB}\right)^6}f_{\text{dmp},6}(r_{AB}) + s_8\frac{C_8^{AB}}{\left(r_{AB}\right)^8}f_{\text{dmp},8}(r_{AB}).\label{eqn:Edisp2}
\end{align}

Using only the second term in equation~\eqref{eqn:Edisp2} is equal to the
second generation dispersion correction D2.\footnote{The calculation of $C_6$
parameters is carried out differently for D2.} $s_6$ and $s_8$ are functional
specific parameters that need to be adjusted for each different \ac{DFT}
functional. The damping functions $f_{\text{dmp},6}$ and $f_{\text{dmp},8}$ are
necessary to cut off the interaction at long distances. The $C_6^{AB}$
dispersion coefficient is calculated by averaging over the dipole
polarisabilities $\alpha$ of the hydrides of the elements $A$ and $B$. The
contributions of the hydrogen atoms have to be corrected for. The value of
$C_6°^{AB}$ can be used to calculate $C_8^{AB}$ and $C_9^{AB}$, which is
contained in the three-body term.

Usually, a zero damping approach is used for the damping function.

\begin{align}
	f_{\text{dmp},n} = \frac{1}{1+6\left( \frac{r_Ar_B}{s_{r,n}r_0^{AB}} \right)^{\alpha_n}}
\end{align}

The name comes from the limit of the damping function which approaches zero with
$r_{AB}$ going to infinity. Alternatively, the Becke-Johnson damping function
can be used, which approaches a constant value with $r_AB\to\infty$ - a more
physically meaningful damping.

