%Introduction

\part{Introduction}
\label{sec:introduction}

%\chapter{Nucleation}
%\label{sec:nucleation}
%
\chapter{Clusters}
\label{sec:clusters}

The term \emph{cluster} is used for a wide range of chemical compounds.
Originally, it was proposed as ``an appropriate one [term] for a finite group of
metal atoms which are held together mainly, or at least to a significant extent,
by bonds directly between the metal atoms, even though some non-metal atoms may
also be intimately associated with the
cluster''.\autocite{Cotton_MetalAtomClusters_1964} However, this definition
limits itself only to the fraction of metal atoms in the periodic table and the
term is not necessarily used in this form today. The most accurate definition of
a cluster is perhaps given through size, as almost any chemical compound with a
finite size of $2$--$10^n$ ($n\lessapprox 7$) atoms is referred to as a
cluster.\autocite{Johnston_Atomicmolecularclusters_2002,Wales_Energylandscapes_2003}
Therefore, clusters are structures, that are of intermediate size, bridging the
gap between small molecules and bulk solids. Thus, they appear naturally when
discussing nucleation phenomena and nano-particles.

Clusters can be divided into several groups that are characterised by the type
of atoms comprising the cluster and therefore its electronic bonding situation.
For example, molecular clusters, which, due to their closed electronic shells,
mainly interact inter-molecularly via weak van-der-Waals forces. However, the
intra-molecular interactions are usually of covalent nature. Such clusters are
found for simple molecules, such as water,\autocite{Liu_WaterClusters_1996}
ammonia\autocite{Beu_Structureammoniaclusters_2001} or carbon
dioxide.\autocite{Takeuchi_GeometryOptimizationCarbon_2008} Moreover,
semi-conductor clusters are bound much more strongly by covalent interactions.
Their name stems from the type of atoms that make up the cluster as they are
semi-conductors in the solid state. Most famously, this group includes carbon
fullerenes\autocite{Kroto_stabilityfullerenesCn_1987}, but also other
semi-conductors like silicon\autocite{Zhu_Structuresstabilitiessmall_2003a} and
germanium.\autocite{Pacchioni_Silicongermaniumclusters_1986} If a cluster is not
monoatomic and the difference in the electronegativity of the atoms is large
enough the covalent bonding situation can change to ionic.

In this thesis, two other types of clusters are investigated, monoatomic metal
and rare gas clusters. The bonding situation in metal clusters is particularly
interesting, because of the high degree of delocalisation and non-directional
bonding. To describe this situation, several bonding models have been developed.
The most simple one is perhaps the \emph{liquid drop model} which approximates
the metal cluster as a uniform conducting sphere, i.e. it is a classical
electrostatic model. The liquid drop model does not give rise to an electronic
structure, which is resolved in the \textit{spherical jellium model}. In this
model the cluster is modelled as a uniform, positively charged sphere filled
with an electron gas, which is solved using the \textit{Schr\"odinger equation}.
This gives rise to quantised electron energy levels and therefore an electronic
shell structure.

Rare gas clusters can form at very low temperatures, when the average kinetic
energy of the rare gas atoms is smaller than the weak dispersive forces between
them. The reason they interact so weakly is because of their closed shell
electronic structures, allowing for neither covalent nor ionic bonding. As
dispersive interactions are a correlation effect of the electrons, it is
difficult to describe them accurately with quantum chemical methods. However,
the interaction can be approximated by simple models like, for example, the
London formula.
%
\begin{align}
    V_\text{disp}=-\frac{C_6}{r^6},\quad C_6=\frac{3\alpha^2I}{4\left(4\pi\varepsilon_0\right)^2}
\end{align}
Here, $I$ is the ionisation potential and $\alpha$ is the atomic polarisability.
In combination with a term describing the repulsive contribution to the energy,
the Lennard-Jones potential can be derived, which is in good agreement on
structure and energy with experimental results for rare gas clusters.

The reason why clusters are studied so extensively lies within their broad range
of properties due to the various different sizes. In smaller clusters,
interesting quantum effects can be observed, which gives rise to the whole field
of nano science. This field, which was most famously inspired by an article by
Richard Feynman,\autocite{Feynman_TherePlentyRoom_1960} is thriving today.
Because clusters are finite objects, they have a boundary that interfaces with
their environment. A similar situation can be found in bulk surfaces, as the
surface atoms (or atoms at the boundary in case of clusters) have lower
coordination numbers than bulk atoms. This under-coordination gives rise to
interesting chemical reactivity and in some case even catalytic activity.

As the clusters grow in size, their properties become more and more bulk-like,
because the electronic states are no longer quantised but quasi-continuous.
Understanding these growth behaviours could help to unravel questions around
crystal growth or at which size metallic conductivity can be first observed. 
\\\newline
In this thesis, three projects related to cluster science, are investigated.
First, the theoretical background required to carry out the calculations will be
reviewed and a quick overview of the program package \textsc{Spheres} developed
in this thesis will be given. The final three sections contain the results of
the investigations. A special type of hollow gold clusters will be investigated
in chapter~\ref{sec:goldendualfullerenes} with respect to their geometry and
stability. These clusters show an interesting connection to the carbon
fullerenes. Rare gas clusters are subject of
chapters~\ref{sec:fromstickyhardspheretoLJtypeclusters} and
\ref{sec:thegregorynewtonclusters}. The investigations in these chapters are
focused on clusters interacting via interaction potentials that describe
dispersive effects. The last chapter also uses graph theoretical methods to
investigate the connectivity of the clusters with respect to the Icosahedron.

