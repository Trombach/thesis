%Abstract

\chapter*{Abstract}
\addcontentsline{toc}{chapter}{Abstract}

%The structures and energies of different types of clusters are studied
%computationally. For this, a combination of graph theoretical methods,
%approximate interaction potentials and accurate quantum chemical calculations is
%used to investigate gold clusters, Lennard-Jones (LJ) clusters and clusters
%bound by the sticky-hard-sphere (SHS) interaction potential.

The structures and stabilities of hollow gold clusters are investigated by means
of density functional theory (DFT) as topological duals of carbon fullerenes.
Fullerenes can be constructed by taking a graphene sheet and wrapping it around
a sphere, which requires the introduction of exactly 12 pentagons. In the dual
case, a (111) face-centred cubic (fcc) gold sheet can be deformed in the same
way, introducing 12 vertices of degree five, to create hollow gold nano-cages.
This one-to-one relationship follows trivially from Euler's polyhedral formula
and there are as many golden dual fullerene isomers as there are carbon
fullerenes. Photoelectron spectra of the clusters are simulated and
compared to experimental results to investigate the possibility of detecting
other dual fullerene isomers. The stability of the hollow gold cages is compared
to compact structures and a clear energy convergence towards the (111) fcc sheet
of gold is observed. 

The relationship between the Lennard-Jones (LJ) and sticky-hard-sphere (SHS)
potential is investigated by means of geometry optimisations starting from the
SHS clusters. It is shown that the number of non-isomorphic structures
resulting from this procedure depends strongly on the exponents of the LJ
potential. Not all LJ minima, that have been discovered in previous work, can
be retrieved this way and the mapping from the SHS to the LJ structures is
therefore non-injective and non-surjective. The number of missing structures is
small and they correspond to energetically unfavourable minima on the energy
landscape. The optimisations are also carried out for an extended Lennard-Jones
potential derived from coupled-cluster calculations for the xenon dimer, and,
although the shape of the potential is not too different from a regular (6,12)-LJ
potential, the number of minima increases substantially.

Gregory-Newton clusters, which are clusters where 12 spheres surround and touch
a central sphere, are obtained from the complete set of SHS clusters. All 737
structures result in an icosahedron, when optimised with a (6,12)-LJ potential.
Furthermore, the contact graphs, consisting only of atoms from the outer shell
of the clusters, are all edge-induced sub-graphs of the icosahedral graph. For higher
LJ exponents the symmetry of the potential energy surface breaks away from the
icosahedral motif towards the SHS landscape, which does not support a perfect
icosahedron for energetic reasons. This symmetry breaking is
mainly governed by the shape of the potential in the repulsive region, with the
long-range attractive region having little influence.



