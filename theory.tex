%theory

%\part{Theory}
%\label{sec:theory}

\chapter{Graph Theory}
\label{sec:graphtheory}

\chapter{Basics of Quantum Chemistry}
\label{sec:basicsofQC}

In this part of the thesis knowledge of standard methods used in quantum
chemistry is required. The following chapters introduce the fundamentals of
quantum mechanics and their application to chemistry related questions. Most of
it is based on standard literature by
Jensen\autocite{Jensen_IntroductionComputationalChemistry_2007}, Szabo and
Ostlund\autocite{Szabo_ModernQuantumChemistry_1996} and Holthausen and
Koch\autocite{Koch_ChemistGuideDensity_2001}.

\section{The Schr\"odinger Equation}
\label{sec:schrodingerequation}

The beginning of the 20th centuary was a very important time for modern
theoretical sciences. Discoveries like Planck's energy quantisation based on
black body radiation\autocite{Planck_UeberGesetzEnergieverteilung_1901} or the
discovery of the wave particle dualism by de
Broglie\autocite{Broglie_RecherchestheorieQuanta_1925} lead to a complete
reformulation of the physical laws governing the smallest of particles. Erwin
Schr\"odinger established a framework based on Hamiltonian mechanics that set
the wave function $\Psi$ at the center of
attention.\autocite{Schrodinger_QuantisierungalsEigenwertproblem_1926} It
contains all information about the system and its evolution in time. For
ground-state calculations it is usually sufficient to look at solutions to the
time-independent Schr\"odinger equation.%
%
\begin{equation}
	\mathbf{H}\Psi=E\Psi\label{eqn:SchrodingerEquation}
\end{equation}%
%
In this eigenvalue equation the Hamilton operator $\mathbf{H}$ acts on the wave
function which results in a solution for the total energyi $E$ of the system. 

The wave function is a function of all spatial $\mathbf{r}$ and spin
coordinates $\omega$ of all the particles in the system. The combination of
these coordinates will be denoted $\mathbf{x}$ on the following pages.  
%
\begin{align}
	\mathbf{x}=\{\mathbf{r},\omega\}
\end{align}
%
The square of the wave function is usually interpreted as a probability density
which is used to normalise it such that the probability of finding an electron
anywhere in space is one. This interpretation is also referred to as Born's
interpretation of the wave
function.\autocite{Born_ZurQuantenmechanikStossvorgaenge_1926,Born_AdiabatenprinzipQuantenmechanik_1927}
%
\begin{align}
	\int |\Psi(\mathbf{x})|^2\dd \mathbf{x}_1\dd \mathbf{x}_2\ldots\dd \mathbf{x}_N=\braket{\Psi(\mathbf{x})}=1
\end{align}
%
The information contained in the wave function can be accessed by an operator
$\mathbf{O}$ acting on the wave function and forming an eigenvalue equation. In
equation~\eqref{eqn:SchrodingerEquation} this operator is the Hamilton
operator, but it could be any operator connected to a physical observable.
Generally, the expectation value of any operator $\expval{\mathbf{O}}$ is
defined as
%
\begin{equation}
    \expval{\mathbf{O}}=\frac{\expval{\mathbf{O}}{\Psi(\mathbf{x})}}{\braket{\Psi(\mathbf{x})}},
\end{equation}
%
or for normalised wave functions
%
\begin{align}
    \expval{\mathbf{O}}=\expval{\mathbf{O}}{\Psi(\mathbf{x})}.
\end{align}
%
For a system of $N$ electrons and $M$ nuclei the non-relativistic Hamilton
operator in atomic units has the form
%
\begin{align}
\begin{aligned}
    \mathbf{H}&=\mathbf{T}_\text{e} + \mathbf{V}_\text{ee} + \mathbf{T}_\text{n} + \mathbf{V}_\text{nn} + \mathbf{V}_\text{en} \\
    &=-\sum_{i=1}^N\frac{1}{2}\nabla_i^2
    + \sum_{i=1}^N\sum_{j>i}^NV_{ij}
    - \sum_{A=1}^M\frac{1}{2}\nabla_A^2
    + \sum_{A=1}^M\sum_{B>A}^MV_{AB}
    + \sum_{i=1}^N\sum_{A=1}^MV_{iA}.\label{eqn:hamiltonoperator}
\end{aligned}
\end{align}
%
The operator contains the kinetic energy of the electrons
$\mathbf{T}_\text{e}$, the electron repulsion $\mathbf{V}_\text{ee}$, the
kinetic energy of the nuclei $\mathbf{T}_\text{n}$, the nucleon repulsion
$\mathbf{V}_\text{nn}$ and the electron-nucleus attraction
$\mathbf{V}_\text{en}$.

\section{The Born-Oppenheimer Approximation}
\label{sec:bornoppenheimerapproximation}

For quantum chemical applications the coupling of the movement of the electrons
and nuclei is usually neglected. This is possible because the atomic mass $m_A$
is so much greater than the electronic mass $m_i$, resulting in much lower
velocities for nuclei compared to
electrons.\autocite{Born_ZurQuantentheorieMolekeln_1927} Therefore, the
electrons can be considered to be moving in a static field of nuclei, meaning
the nuclear kinetic term $\mathbf{T}_\text{n}$ can be neglected and the nuclear
repulsion term $\mathbf{V}_\text{nn}$ becomes a constant. The resulting
electronic Hamilton operator $\mathbf{H}_\text{e}$ describes electrons moving
in a field of positive point charges.
%
\begin{align}
    \mathbf{H}_\text{e}=\mathbf{T}_\text{e} + \mathbf{V}_\text{ee} + \mathbf{V}_\text{en}
\end{align}
%
Solving the Schr\"dinger equation for this operator yields the electronic wave
function $\Psi_\text{e}$. It depends explicitly on the electronic coordinates,
but only parametrically on the nuclear coordinates, therefore spanning a
potential energy surface upon which the nuclei move. To get the total energy of
the system the constant repulsion between the nuclei has to be added to the
electronic operator. 
%
\begin{align}
    \mathbf{H}=\mathbf{H}_\text{e}+V_\text{nn}\label{eqn:hamiltonoperatorfinal}
\end{align}
%
\section{The Hartree-Fock Approximation}
\label{sec:hartreefockapproximation}

An exact solution to the Schr\"odinger equation in the boundary conditions of
the Born-Oppenheimer approximation for systems containing more than one
electron is not possible. The solution has to be approximated by appropriate
methods; one possible solution are the Hartree-Fock equations. This set of
equations determines the energy of a wave function expressed as a Slater
determinant
$\Phi_\text{SD}$.\autocite{Slater_TheoryComplexSpectra_1929,Fock_NaeherungsmethodezurLoesung_1930}
%
\begin{equation}
     \Phi_{SD}=\frac{1}{\sqrt{N!}}
     \begin{vmatrix}
         \phi_1(\mathbf{x}_1) & \phi_2(\mathbf{x}_1) & \cdots & \phi_N(\mathbf{x}_1)\\
         \phi_1(\mathbf{x}_2) & \phi_2 (\mathbf{x}_2) & \cdots & \phi_N(\mathbf{x}_2)\\
         \vdots & \vdots & \ddots & \vdots\\
         \phi_1(\mathbf{x}_N) & \phi_2(\mathbf{x}_N) & \cdots & \phi_N(\mathbf{x}_N)
     \end{vmatrix}
     \label{eqn:SlaterDet}
\end{equation}
%
Here, the $\phi_i$ denote one electron spin orbitals and
$\mathbf{x}_i=\{\mathbf{r}_i,\omega_i\}$ are spatial ($\mathbf{r}_i$) and spin
($\omega_i$) coordinates of the electrons. A Slater determinant obeys the Pauli
exclusion principle, which requires the electronic wave function to be
anti-symmetric with respect to interchanging the coordinates of two electrons.

In the Hartree-Fock approximation the energy of a single Slater determinant is
used as an approximation for the total energy of the system. As explained
previously, the energy of a wave function can be determined by the action of
the Hamilton operator on the wave function. The Hamilton operator in the
Born-Oppenheimer approximation from equation~\eqref{eqn:hamiltonoperatorfinal}
can be rewritten in terms of one-electron operators $\mathbf{h}_i$.
%
\begin{align}
\begin{aligned}
    \mathbf{H}&=\sum_i^N\left(\mathbf{h}_i + \sum_{j>i}^N\frac{1}{|\mathbf{r}_i-\mathbf{r}_j|}\right) + V_\text{nn} \\
    \mathbf{h}_i&=-\frac{1}{2}\nabla_i^2-\sum_A^M\frac{Z_A}{|\mathbf{r}_A-\mathbf{r}_i|}\label{eqn:hamiltonoperatorhartreefock}
\end{aligned}
\end{align}
%
Here, $Z_A$ denotes the charge of nucleus $A$. $\mathbf{h}_i$ depends only on
the kinetic energy of electron $i$ and its potential energy in the field of all
$M$ nuclei. When $\mathbf{h}_i$ acts on a Slater determinant the result is the
respective matrix element $h_i$. Only parts of the Slater determinant without
permutation of electron coordinates can give a non-zero contribution to the
eigenvalue.
%
\begin{align}
    \begin{aligned}
        h_1&=\mel{\phi_1(\mathbf{x}_1)\phi_2(\mathbf{x}_2)\ldots\phi_N(\mathbf{x}_N)}{\mathbf{h}_1}{\phi_1(\mathbf{x}_1)\phi_2(\mathbf{x}_2)\ldots\phi_N(\mathbf{x}_N)}\\
        &=\mel{\phi_1(\mathbf{x}_1)}{\mathbf{h}_1}{\phi_1(\mathbf{x}_1)}\braket{\phi_2(\mathbf{x}_2)}\ldots\braket{\phi_N(\mathbf{x}_N)} \\
        &=\mel{\phi_1(\mathbf{x}_1)}{\mathbf{h}_1}{\phi_1(\mathbf{x}_1)}
    \end{aligned}
\end{align}
%
The remainder of the Hamilton operator from
equation~\eqref{eqn:hamiltonoperatorhartreefock} is dependant on two electron
coordinates. It is convenient to define a two-electron operator $\mathbf{g}_{ij}$
with the matrix elements $g_{ij}=\frac{1}{|\mathbf{r}_i-\mathbf{r}_j|}$. Its
action on the part of the Slater determinant with no permutation of electron
coordinates results in the Coulomb integral and the corresponding matrix
element $J_{ij}$, which can be interpreted as the classical Coulomb repulsion.
%
\begin{align}
    \begin{aligned}
        J_{12}&=\mel{\phi_1(\mathbf{x}_1)\phi_2(\mathbf{x}_2)\ldots\phi_N(\mathbf{x}_N)}{\mathbf{g}_{12}}{\phi_1(\mathbf{x}_1)\phi_2(\mathbf{x}_2)\ldots\phi_N(\mathbf{x}_N)}\\
        &=\mel{\phi_1(\mathbf{x}_1)\phi_2(\mathbf{x}_2)}{\mathbf{g}_{12}}{\phi_1(\mathbf{x}_1)\phi_2(\mathbf{x}_2)}\ldots\braket{\phi_N(\mathbf{x}_N)} \\
        &=\mel{\phi_1(\mathbf{x}_1)\phi_2(\mathbf{x}_2)}{\mathbf{g}_{12}}{\phi_1(\mathbf{x}_1)\phi_2(\mathbf{x}_2)}
    \end{aligned}
\end{align}
%
As $\mathbf{g}_{ij}$ depends on the coordinates of two electrons it also yields
non-zero matrix elements for parts of the Slater determinant, where two
electron coordinates have been swapped.
%
\begin{align}
    \begin{aligned}
        K_{12}&=\mel{\phi_1(\mathbf{x}_1)\phi_2(\mathbf{x}_2)\ldots\phi_N(\mathbf{x}_N)}{\mathbf{g}_{12}}{\phi_2(\mathbf{x}_1)\phi_1(\mathbf{x}_2)\ldots\phi_N(\mathbf{x}_N)}\\
        &=\mel{\phi_1(\mathbf{x}_1)\phi_2(\mathbf{x}_2)}{\mathbf{g}_{12}}{\phi_2(\mathbf{x}_1)\phi_1(\mathbf{x}_2)}\ldots\braket{\phi_N(\mathbf{x}_N)} \\
        &=\mel{\phi_1(\mathbf{x}_1)\phi_2(\mathbf{x}_2)}{\mathbf{g}_{12}}{\phi_2(\mathbf{x}_1)\phi_1(\mathbf{x}_2)}
    \end{aligned}
\end{align}
%
$K_{ij}$ is called the exchange integral and has no classical interpretation.
As swapping coordinates in the Slater determinant changes its sign, the result
of the exchange integral has a negative sign. The total energy of the system is
now given by the sum over all integrals described above.
%
\begin{align}
    E=\sum_{i=1}^Nh_i + \frac{1}{2}\sum_{i=1}^N\sum_{j=1}^N(J_{ij}-K_{ij})+V_\text{nn}\label{eqn:HFenergy}
\end{align}
%
Defining operators $\mathbf{J}_i$ and $\mathbf{K}_i$ for Coulomb and exchange
integral equation~\eqref{eqn:HFenergy} becomes
%
\begin{align}
    E=\sum_{i=1}^N\mel{\phi_i}{\mathbf{h}_i}{\phi_i} + \frac{1}{2}\sum_{i=1}^N\sum_{j=1}^N \left( \mel{\phi_i}{\mathbf{J}_j}{\phi_i} - \mel{\phi_i}{\mathbf{K}_j}{\phi_i} \right) + V_\text{nn}\\
    \begin{aligned}
        \mathbf{J}_i\ket{\phi_j(\mathbf{x}_j)}&=\mel{\phi_i(\mathbf{x}_i)}{\mathbf{g}_{ij}}{\phi_i(\mathbf{x}_i)}\ket{\phi_j(\mathbf{x}_j)}\\
        \mathbf{K}_i\ket{\phi_j(\mathbf{x}_j)}&=\mel{\phi_i(\mathbf{x}_i)}{\mathbf{g}_{ij}}{\phi_j(\mathbf{x}_i)}\ket{\phi_i(\mathbf{x}_j)}.
    \end{aligned}\label{eqn:HFenergyoperator}
\end{align}
%
For the purpose of quantum chemical calculations the energy of an arbitrary
Slater determinant is usually not useful. More interesting, however, is to find
the Slater determinant that minimises the energy under the boundary condition
of keeping the orthonormality condition between spin orbitals. In other words,
we try to find the derivative of equation~\eqref{eqn:HFenergyoperator}.
Minimising the energy under external boundary conditions can be solved using
Lagrange multipliers. With their help it is possible to define the Fock
operator $\mathbf{f}_i$, which is an effective one-electron operator.
%
\begin{align}
    \mathbf{f}_i=\mathbf{h}_i + \sum_j^N\left( \mathbf{J}_j - \mathbf{K}_j \right)
\end{align}
%
The action of the Fock operator on an element of the Slater determinant yields
the Hartree-Fock equations.
%
\begin{align}
    \mathbf{f}_i\ket{\phi_i}=\sum_j^N\lambda_{ij}\ket{\phi_j}\label{eqn:HFequations}
\end{align}
%
$\lambda_{ij}$ are Lagrange multipliers remaining from the constrained
minimisation. They can be re-written in matrix form and subsequently
diagonalised by a unitary transformation. This yields the canonical
Hartree-Fock equations.
%
\begin{align}
    \mathbf{f}_i\ket{\phi'_i}=\sum_j^N\varepsilon_i\ket{\phi'_j}\label{eqn:HFequationscanonical}
\end{align}
%
$\varepsilon_i$ are orbital energies of the electrons. According to Koopman's
theorem they can be interpreted as ionisation energies for occupied and
electron affinities for unoccupied states. The Fock operator depends on all
occupied states, making it a pseudo eigenvalue equation. Hence, solutions have
to be found iteratively starting from and arbitrary set of orbitals. After a
set of convergence criteria has been met, the effective potential is said to
remain unchanged, creating a \ac{SCF} solution.  \\\newline Solving the
canonical Hartree-Fock equations for larger systems is possible, but costly.
Usually they are solved using a basis set expansion to approximate the unknown
molecular orbitals. The basis functions are usually chosen to agree with the
underlying physics of the system. For example, periodic plane waves are
usually used when periodic boundary conditions are required. For calculations
in the gas phase the basis functions are usually exponential functions
centered at the nuclei. In this case the approximation is called \ac{LCAO}.
Technically, a basis set expansion is not an approximation, but as one is
limited to a finite amount of basis functions $P$ the expansion does not give
an exact expression for a molecular orbital $\phi_i$. For a set of $\alpha$
basis functions $\chi_\alpha$ the expansion can be expressed as follows.
%
\begin{align}
    \phi_i=\sum_\alpha^Pc_{\alpha i}\chi_\alpha
\end{align}
%
This leads to the Hartree-Fock equations expressed in the basis set
approximation.
%
\begin{align}
    \begin{aligned}
        \mathbf{f}_i\sum_\alpha^Pc_{\alpha i}\chi_\alpha&=\varepsilon_i\sum_\alpha^Pc_{\alpha i}\chi_\alpha \\
        \sum_\alpha^Pc_{\alpha i}\underbrace{\mel{\chi_\alpha}{\mathbf{f}_i}{\chi_\beta}}_{F_{\alpha\beta}}&=\varepsilon_i\sum_\alpha^Pc_{\alpha i}\underbrace{\braket{\chi_\alpha}{\chi_\beta}}_{S_{\alpha\beta}}
    \end{aligned}
\end{align}
%
These are the Roothaan-Hall
equations\autocite{Roothaan_NewDevelopmentsMolecular_1951,Hall_molecularorbitaltheory_1951}
which are usually written in matrix form.
%
\begin{align}
    \mathbf{FC}=\mathbf{SC\varepsilon}
\end{align}
%
$\mathbf{F}$ is the Fock matrix, $\mathbf{S}$ is the overlap matrix and
$\mathbf{C}$ contains the orbital coefficients. These equations have to be
solved iteratively and to reduce computational cost the first step usually
involves calculating a density matrix $\mathbf{D}$.
%
\begin{align}
    \mathbf{D}=\sum_{i}^\text{occ. MO} c_{\mu i}c_{\nu i}
\end{align}
%
$\mathbf{D}$ can be used to generate a Fock matrix, which will be diagonalised
yielding a new set of orbital coefficients. These will be used to generate a
new generation of the density matrix. This procedure will be repeated until
the coefficients of the new generation are equal (up to a certain precision)
to the ones of the parent generation. This marks the end of the \ac{SCF} cycle.

\section{Density Functional Theory}
\label{sec:dft}

The Hartree-Fock method belongs to the class of mean field approximations,
which means the electrons don't interact directly with each other, but each
electron is moving in a field created by all other electrons. The Hartree-Fock
energy $E_\text{HF}$ is therefore never exact even in the infinite basis set
limit. The difference to the exact energy $E_0$ was first named correlation
energy $E_\text{corr}$ by L\"owdin\autocite{Lowdin_CorrelationProblemManyElectron_1958}.
%
\begin{align}
    E_\text{corr} = E_0 - E_\text{HF}
\end{align}
%
Even though the correlation energy only attributes for about \SI{1}{\percent}
of the total energy it is an important contribution in molecular systems. A
multitude of methods has been developed to treat the correlation energy more
accurately. They can generally be subdivided into post-Hartree-Fock methods and
\ac{DFT}. While post-Hartree-Fock methods rely on a Hartree-Fock wave function
as a starting point, \ac{DFT} is in principle a wave function free method. It
establishes a connection between the energy of the system and the electron
density $\rho$ instead of the wave function. The electron density is observable
and a positive quantity, which makes \ac{DFT} easier to grasp than wave
function based methods. However, the electron density is related to the wave
function via its square. 
%
\begin{align}
    \rho(\mathbf{r}_1)=N\int\dots\int|\Psi(\mathbf{x}_1,\mathbf{x}_2,\dots,\mathbf{x}_N)|^2\;\dd\omega_1\dd\mathbf{x}_2\dots\dd\mathbf{x}_N\label{eqn:electrondensity}
\end{align}
%
Equation~\eqref{eqn:electrondensity} describes the probability density of
finding one of the $N$ electrons in the volume $\dd \mathbf{r}_1$ with
arbitrary spin. All other $(N-1)$ electrons occupy any volume elements, but
have opposite spin. As electrons are indistinguishable the probability to find
any of the $N$ electrons in the volume $\dd\mathbf{r}_1$ is equal to $N$ times
the probability of finding a specific electron in that volume.  A justification
for using the electron density instead of the wave function was found by
Hohenberg and Kohn in 1964.\autocite{Hohenberg_InhomogeneousElectronGas_1964}

\subsection{Hohenberg-Kohn Theorems}
\label{sec:HKtheorems}

The first Hohenberg-Kohn theorem implies that the electron density defines a
unique external potential that contains all information about the system. If
two external potentials are different they cannot lead to the same ground state
electron density. From this, the total energy of a system can be expressed as follows.
%
\begin{align}
    E_0\left[ \rho_0(\mathbf{r}) \right] = T\left[ \rho_0(\mathbf{r}) \right] + E_\text{ee} \left[ \rho_0(\mathbf{r}) \right] + E_\text{Ne} \left[ \rho_0(\mathbf{r}) \right]\label{eqn:energyDFT}
\end{align}
%
The aforementioned external potential corresponds to $E_\text{Ne} \left[
	\rho_0(\mathbf{r}) \right]$, which is the only system-dependent term of
equation~\eqref{eqn:energyDFT}. The system-independent terms for the kinetic
energy $T\left[ \rho_0(\mathbf{r}) \right]$ and the electron-electron
interaction $E_\text{ee} \left[ \rho_0(  \mathbf{r}) \right]$ can be combined
to the Hohenberg-Kohn functional $F_\text{HK}\left[ \rho_0(
\mathbf{r})\right]$.
%
\begin{align}
    E_0\left[ \rho_0(\mathbf{r}) \right] = \int\dd\mathbf{r}\rho_0(\mathbf{r})V_\text{Ne}+ F_\text{HK}\left[ \rho_0(\mathbf{r})\right]\label{eqn:energyDFTHK}
\end{align}
%
Would the exact Hohenberg-Kohn functional be known
equation~\eqref{eqn:energyDFTHK} would be an exact solution to the
Schr\"odinger equation. In practice, there are no exact expressions for
$T\left[ \rho_0(\mathbf{r}) \right]$ or $E_\text{ee} \left[ \rho_0(
\mathbf{r}) \right]$ knwon. However, the latter can be expressed in terms of a
classical Coulomb term $J\left[\rho\right]$ and a non-classical energy contribution
$E_\text{nc}\left[\rho\right]$.
%
\begin{align}
    E_\text{ee} \left[ \rho \right]=J\left[\rho\right]+E_\text{nc}\left[\rho\right]=\frac{1}{2}\iint\frac{\rho(\mathbf{r}_1)\rho(\mathbf{r}_2)}{r_{12}}\dd\mathbf{r}_1\dd\mathbf{r}_2+E_\text{nc}\left[\rho\right]
\end{align}
%
The second Hohenberg-Kohn theorem warrants that a trial energy density
$\widetilde{\rho}$ always yields an energy greater or equal to the exact ground
state energy.
%
\begin{align}
    E_0 \leq E\left[\widetilde{\rho}\right]
\end{align}
%
It is equivalent to the variational theorem\footnote{The variational theorem
states that no trial wave function can result in a smaller energy than the
exact ground state wave function.} in wave function theory. However, it is only
valid for the exact Hohenberg-Kohn functional, which is unknown. A solution to
this problem was given in 1965 by Kohn and
Sham.\autocite{Kohn_SelfConsistentEquationsIncluding_1965}

\subsection{Kohn-Sham Theory}
\label{sec:kohnshamtheory}

One of the biggest flaws in orbital free \ac{DFT} ist the poor description of
the kinetic energy term. Kohn and Sham realised it would be easier to describe
it in terms of a reference system of non-interacting electrons. Their kinetic
energy $T_S$ can be expressed in terms of a Slater determinant consisting of
one-electron orbitals $\phi_i$ also called Kohn-Sham orbitals.
%
\begin{align}
    T_S=-\frac{1}{2}\sum_i^N\mel{\phi_i}{\nabla_i^2}{\phi_i}
\end{align}
%
The electron density resulting from the Kohn-Sham orbitals is required to be
equal to the density of the real system.
%
\begin{align}
    \rho_S(\mathbf{r}) = \rho_0(\mathbf{r})
\end{align}
%
For such a system the Coulomb interaction between electrons and nuclei can be
calculated exactly. The only unknown remaining terms are the non-classical
contribution to the electron-electron interaction $E_\text{nc}$ and a
contribution to the kinetic energy because of electron correlation $T_C$. These
terms can be combined to the exchange-correlation term $E_{XC}$.
%
\begin{align}
    E\left[\rho\right]=T_S\left[\rho\right] + J\left[\rho\right] + E_{XC}\left[\rho\right] + E_\text{Ne}\left[\rho\right] \\
    \begin{aligned}
        =&-\frac{1}{2}\sum_i^N\mel{\phi_i}{\nabla_i^2}{\phi_i} + \frac{1}{2}\sum_i^N\sum_j^N\iint \dd\mathbf{r}_1\dd\mathbf{r}_2|\phi_i(\mathbf{r}_1)|^2\frac{1}{r_{12}}|\phi_j(\mathbf{r}_2)|^2 \\
        &+ E_{XC}\left[\rho\right] - \sum_i^N\int\dd\mathbf{r}_1\sum_A^M|\phi_i(\mathbf{r}_1)|^2
    \end{aligned}\label{eqn:energyKS}
\end{align}
%
Similar to Hartree-Fock theory, the minimal energy can be calculated using
Lagrange multipliers. The potential terms from equation~\eqref{eqn:energyKS}
can be combined to an effective potential $V_S$, which allows for the
definition of a Kohn-Sham operator $\mathbf{f}_\text{KS}$ analogous to the Fock
operator in Hartree-Fock theory.
%
\begin{align}
    V_S(\mathbf{r}_1) = \int\dd\mathbf{r}_2\frac{\rho(\mathbf{r}_2)}{r_{12}} + V_{XC}(\mathbf{r}_1) - \sum_A^M\frac{Z_A}{r_{1A}}\\
    \mathbf{f}_\text{KS} = -\frac{1}{2}\nabla^2+V_S(\mathbf{r}_1)\\
    \mathbf{f}_\text{KS}\phi_i=\varepsilon_i\phi_i
\end{align}
%
These are the Kohn-Sham equations and they have to be solved in an iterative
procedure, because of the Kohn-Sham operator depending on the occupied
orbitals.  The unknown Kohn-Sham orbitals are usually expanded in terms of
basis functions such that the equations can be expressed in matrix form,
similar to the Roothaan-Hall equations. The Fock matrix is replaced by the
Kohn-Sham matrix $\mathbf{F}_\text{KS}$.
%
\begin{align}
    \mathbf{F}_\text{KS}\mathbf{C} = \mathbf{SC\varepsilon}
\end{align}
%
\subsection{Exchange and Correlation Functionals}
\label{sec:exchangecorrelationfunctionals}

The key to solving the Kohn-Sham equations is the exchange-correlation energy
$E_\text{XC}$. Over the years there have been lots of proposals for its form,
the oldest being the \ac{LDA}. It is based on the uniform electron gas for
which analytical functionals for exchange and correlation are known.
%
\begin{align}
    E_{XC}^\text{LDA}\left[\rho\right]=\int\rho(\mathbf{r})\varepsilon_{XC}\left[\rho(\mathbf{r})\right]\dd\mathbf{r}
\end{align}
%
The exchange-correlation energy $\varepsilon_{XC}\left[\rho(\mathbf{r})\right]$
is weighted with the probability of finding and electron at this point in
space. After separation of the exchange and correlation parts the exchange
energy can be described by a term developed by Slater.
%
\begin{align}
	\varepsilon_{XC}\left[\rho(\mathbf{r})\right] = \varepsilon_{X}\left[\rho(\mathbf{r})\right] + \varepsilon_{C}\left[\rho(\mathbf{r})\right]\\
	E_{X}^\text{LDA}\left[\rho\right] = -C_X\int\rho(\mathbf{r})^{\frac{4}{3}}\dd\mathbf{r}
\end{align}
%
No simple formula for the correlation term is known.

\ac{LDA} describes the inhomogeneous electron density by dividing it up into
small homogeneous volumes. An improvement over \ac{LDA} can be made if the
homogeneous electron density is expanded in a Taylor series. Truncating after the
first term gives the \ac{LDA} approximation, including one more term is called
the \ac{GEA}. Because \ac{GEA} doesn't correctly describe the
exchange-correlation hole function it gives worse results that \ac{LDA}.

A break-through for theoretical chemistry has been achieved with the
introduction of the \ac{GGA}. It uses the \ac{GEA} hole functions and tailors
them to physically meaningful boundary conditions. 

In this work the PBE functional by Perdew, Burke and Ernzerhof\autocite{Perdew_GeneralizedGradientApproximation_1996,PerdewGeneralizedGradientApproximation1997},
which belongs to the group of \ac{GGA} functionals, was used for most
calculations. They published both, correlation and exchange expressions for
this functional.
%%show functional

\subsection{Dispersion Corrections}
\label{sec:dispersioncorrections}

Long-range dispersive effects are part of the correlation energy and most
\ac{DFT} functionals can only describe these effects to very limited degree.
Grimme \textit{et al.} developed a method that can be used in conjunction with
most density functionals.\autocite{Grimme_consistentaccurateinitio_2010} Based
on the functional used it calculates a dispersive energy contribution
$E_\text{disp}$ (and gradient contribution for optimisations) that can be added
to the \ac{DFT} energy $E_\text{DFT}$. The dispersion energy is always of
attractive nature and therefore has a negative sign by convention.
%
\begin{align}
	E = E_\text{DFT} + E_\text{disp}
\end{align}
%
In the third generation D3 dispersion correction, which was used in this work,
the calculation of $E_\text{disp}$ involves solving a two- and three-body term.
The two-body term $E^{(2)}$ is more important and is only a function of the
distance between two nuclei $r_{AB}$.
%
\begin{align}
	E^{(2)}=\sum\limits_{AB}s_6\frac{C_6^{AB}}{\left(r_{AB}\right)^6}f_{\text{dmp},6}(r_{AB}) + s_8\frac{C_8^{AB}}{\left(r_{AB}\right)^8}f_{\text{dmp},8}(r_{AB}).\label{eqn:Edisp2}
\end{align}
%
Using only the first term in equation~\eqref{eqn:Edisp2} is equal to the
second generation dispersion correction D2.\footnote{The calculation of $C_6$
parameters is carried out differently for D2.} $s_6$ and $s_8$ are functional
specific parameters that need to be adjusted for each different \ac{DFT}
functional. The damping functions $f_{\text{dmp},6}$ and $f_{\text{dmp},8}$ are
necessary to cut off the interaction at long distances. The $C_6^{AB}$
dispersion coefficient is calculated by averaging over the dipole
polarisabilities $\alpha$ of the hydrides of the elements $A$ and $B$. The
contributions of the hydrogen atoms have to be corrected for. The value of
$C_6^{AB}$ can be used to calculate $C_8^{AB}$ and $C_9^{AB}$, which is
contained in the three-body term.

Usually, a zero damping approach is used for the damping function.
%
\begin{align}
	f_{\text{dmp},n} = \frac{1}{1+6\left( \frac{r_Ar_B}{s_{r,n}r_0^{AB}} \right)^{\alpha_n}}
\end{align}
%
The name comes from the limit of the damping function which approaches zero
with $r_{AB}$ going to infinity. Alternatively, the Becke-Johnson damping
function\autocite{Grimme_Effectdampingfunction_2011} can be used, which
approaches a constant value with $r_{AB}\to\infty$ - a more physically
meaningful damping.

\section{Basis Sets}
\label{sec:basissets}

Both the Hartree-Fock approximation and \ac{DFT} are usually solved in a basis
set expansion. For molecules in the gas phase a atom-centered approach is
usually the method of choice. Depending on the nature of the system it can be
beneficial to use a different approach like plane waves which are often used in
periodic boundary calculations.

The most commonly type of basis functions used in molecular calculations are
\acp{GTO} and \acp{STO} and mainly differ by the computational cost associated
with them. \acp{STO} are derived from the exact solutions to the hydrogen atom
and therefore depend parametrically on the three quantum numbers $n$,
$l$, $m$.
%
\begin{align}
    \chi_{\zeta_S,n,l,m}(r,\theta,\varphi) = NY_{l,m}(\theta,\varphi)r^{n-1}e^{-\zeta_S r}
\end{align}
%
The functions are split up into a radial part only depending on the spherical
coordinate $r$ and the orbital coefficient $\zeta_S$ and an angular part
$Y_{l,m}(\theta,\varphi)$ which are spherical harmonics only depending on the
angles $\theta$ and $\varphi$. \acp{STO} describe the cusp and the exponential
decay in the core region well, but don't have analytical solutions for three-
or four-center integrals. They are usually only used when very high precision
is required.

\acp{GTO} use the same spherical harmonics to describe the angle dependent
part, but contain a Gaussian function to describe the $r$-dependant term.
%
\begin{align}
    \begin{aligned}
    \chi_{\zeta_G,n,l,m}(r,\theta,\varphi) &= NY_{l,m}(\theta,\varphi)r^{2n-2-l}e^{\zeta_Gr^2}\\
    \chi_{\zeta_G,l_x,l_y,l_z}(x,y,z) &= N x^{l_x}y^{l_y}z^{l_z}e^{-\zeta r^2}
    \end{aligned}
\end{align}
%
In the core region the function is continuous and its derivative is zero and
therefore it gives a worse description of the system. However, it is possible
to combine multiple \acp{GTO} in a linear combination to approximate the shape
of the \ac{STO}.
%
\begin{align}
    \chi^\text{STO} = \sum_ia_i\chi_i^\text{GTO}
\end{align}
%
In addition to this expansion it is typical to use more than one
function per atomic orbital to improve the flexibility of the basis set and
therefore the description of the orbital. This is done by choosing different
values for the variable $\zeta$ and combining them to one orbital. Basis sets
with two basis functions per orbital are called "double zeta" basis sets, three
basis functions yield a "triple zeta" basis set \textit{etc}.

The inner electrons have the largest contribution to the total energy of a
molecule. However, for questions about chemical reactivity or catalytic
activity the valence regions are of far greater importance than the core
regions. It is therefore beneficial to use split-valence basis sets that use
more basis functions to describe the valence region and basis set contractions
in the core regions. This usually leads to a smaller computational cost without
loosing accuracy. 

\section{Effective Core Potentials}
\label{sec:ECP}

\section{Periodic Boundary Conditions}
\label{sec:PBC}

\chapter{Geometry Optimisation}
\label{sec:geometryoptimisation}

Whether one uses quantum chemical methods or simple two-body potentials to
investigate the properties of molecules and clusters, finding the coordinates
of all atoms that minimise the chosen energy function is always of importance.
Methods for geometry optimisation have been used in all parts of this thesis
and will therefore be explained in detail in the following chapters. If not
mentioned otherwise the theory is based on
Fletcher's\autocite{Fletcher_Practicalmethodsoptimization_1987} introductory
book.

Finding the set of coordinates $\mathbf{x}$ that minimise a given energy
function is an optimisation problem. There are a multitude of methods available
that are all suited for different types of problems. In the following sections
the theory of local minimisation will be discussed in terms of a general
objective function $f$ that can be replaced with any energy function.

\section{General Considerations about Minima}
\label{sec:GeneralRemarksAboutMinima}

The \textit{objective function} $f$ is said to have a minimum value (or simply
\textit{minimum}) at the set of coordinates $\mathbf{x^*}$, which is called a
\textit{minimiser} of the objective function $f$. Usually, optimisation
procedures locate local minimisers, while the problem of finding global
minimisers is considerably more difficult and requires clever algorithms.
General definitions of local minimisers can be formulated in form of
\textit{strict local minimisers} ($f(\mathbf{x})>f(\mathbf{x^*})$) or
\textit{isolated local minimisers} ($\mathbf{x^*}$ is the only local minimiser
in its neighbourhood).

The definition becomes simpler when one only considers smooth functions as the
minimisers can be characterised in terms of first and second derivatives. A
smooth function needs to be continuous and continuously differentiable,
therefore, a vector of first partial derivatives $\nabla
f(\mathbf{x})=\mathbf{g}(\mathbf{x})$ must exist for any $\mathbf{x}$.
%
\begin{equation}
    \nabla f(\mathbf{x})=
    \begin{pmatrix}
        \pdv*{f}{x_1}\\
        \pdv*{f}{x_2}\\
        \vdots\\
        \pdv*{f}{x_n}
    \end{pmatrix}
\end{equation}
%
A twice continuously differentiable function additionally allows for the
definition of a matrix of second partial derivatives $\nabla^2
f(\mathbf{x})=\mathbf{G}(\mathbf{x})$ also called a Hessian matrix.
%
\begin{equation}
    \nabla^2 f(\mathbf{x})=
    \begin{pmatrix}
        \pdv*[2]{f(\mathbf{x})}{{x_1}} & \pdv*{f(\mathbf{x})}{x_1}{x_2} & \ldots & \pdv*{f(\mathbf{x})}{x_1}{x_{n}}\\
        \pdv*{f(\mathbf{x})}{x_2}{x_1} & \pdv*[2]{f(\mathbf{x})}{{x_2}} & \ldots & \pdv*{f(\mathbf{x})}{x_2}{x_{n}}\\
        \vdots & \vdots & \ddots & \vdots\\
        \pdv*{f(\mathbf{x})}{x_{n}}{x_1} & \pdv*{f(\mathbf{x})}{x_{n}}{x_2} & \ldots & \pdv*[2]{f(\mathbf{x})}{{x_{n}}}
    \end{pmatrix}
\end{equation}
%
Most interatomic potentials have smooth potential energy landscapes, which
justifies this simplification. 

To derive conditions for a point to be a local minimiser consider any line
through the minimiser $\mathbf{x^*}$
%
\begin{equation}
    \mathbf{x}(\alpha)=\mathbf{x^*}+\alpha\mathbf{s}.
\end{equation}
%
Using the chain rule the derivative $\dv{f}{\alpha}$ can be expressed as
$\dv{f}{\alpha}=\mathbf{s^\text{T}}\nabla
f$.\footnote{$\dv{\alpha}=\sum_i\dv{\alpha}x_i(\alpha)\pdv{x_i}=\sum_i
s_i\pdv{x_i}=\mathbf{s^\text{T}}\nabla$} At $\mathbf{x^*}$ ($\alpha=0$) this
line has a slope of zero and a non-negative curvature, which means
$\mathbf{s^\text{T}}\nabla f(\mathbf{x^*})=0$. Following the same argument for
the second derivative a second condition $\mathbf{s^\text{T}}\nabla^2
f(\mathbf{x^*})\mathbf{s}\geq 0$ can be derived. As these conditions must be
true for all $\mathbf{s}$ we can for example consider the case
$\mathbf{s}=\mathbf{e_1}$, with $\mathbf{e_1}$ being a unit vector, to see that
%
\begin{align}
    \mathbf{g^*}&=0\label{eqn:firstordernecessary}\\
    \mathbf{s^\text{T}}\mathbf{G^*}\mathbf{s}&\geq 0.\label{eqn:secondordernecessary}
\end{align}
%
Note that $\mathbf{g^*}=\mathbf{g}(\mathbf{x^*})$ and
$\mathbf{G^*}=\mathbf{G}(\mathbf{x^*})$ are used to simplify the notation. In
the following sections this short hand notation will also be extended to the
objective function $f$ and general points $\mathbf{x}^{(k)}$, i.e.
$f^{(k)},\mathbf{g}^{(k)},\mathbf{G}^{(k)},\dots=f\left(\mathbf{x}^{(k)}\right),\mathbf{g}\left(\mathbf{x}^{(k)}\right),\mathbf{G}\left(\mathbf{x}^{(k)}\right),\dots$.

Equations~\eqref{eqn:firstordernecessary} and \eqref{eqn:secondordernecessary}
are necessary (but not sufficient) conditions for local minimisers. In fact, as
equation~\eqref{eqn:firstordernecessary} is derived from first order variations
in $f$ it is considered a first order necessary condition, while
equation~\eqref{eqn:secondordernecessary} is considered a second order
necessary condition. It can be shown that sufficient conditions for local
minimisers are equation~\eqref{eqn:firstordernecessary} and
$\mathbf{s^\text{T}}\mathbf{G^*}\mathbf{s}>
0$.\autocite{Fletcher_Practicalmethodsoptimization_1987} The reason for this
minor change for the second order condition is that
equation~\eqref{eqn:secondordernecessary} also holds true for points of zero
curvature. In other words the sufficient conditions for a local minimiser are
the gradient to be zero and the Hessian matrix to be positive definite at
$\mathbf{x^*}$.

In practice, minimisation schemes usually locate $\mathbf{x^*}$ that only
fulfil the first condition $\mathbf{g^*}=0$. As those points can
either refer to minima, maxima or saddle points they are called stationary
points. A located stationary point has to be checked for his character, but it
is usually not feasible to check equation~\eqref{eqn:secondordernecessary} as
it can't be checked numerically. In this work the eigenvalues of
$\mathbf{G^*}$ are used to verify local minimisers, which have to be
positive.

\section{Properties of Optimisation Algorithms}
\label{sec:PropertiesOfOptimisationAlgorithms}

To have any practical usefulness an iterative optimisation algorithm should
obey a few requirements. For instance, the algorithm should move steadily
towards the local minimiser $\mathbf{x^*}$ and converge quickly to a
user-defined convergence criterion. The rate of convergence, decisive for the
performance of the algorithm, can be quantified by defining the error
%
\begin{equation}
    \Delta\mathbf{x}^{(k)}=\mathbf{x}^{(k)}-\mathbf{x^*}.
\end{equation}
%
Here, $\mathbf{x}^{(k)}$ denotes the $k$th iterate with $\mathbf{x}^{(1)}$
referring to the starting point of the iterative procedure. The local
convergence rate can then be given as the fraction of the Euclidean norm
$||\cdot||$ of the errors of consecutive steps.
%
\begin{equation}
    \frac{\left|\left|\Delta\mathbf{x}^{(k+1)}\right|\right|}{\left|\left|\Delta\mathbf{x}^{(k)}\right|\right|^{p}}\leq a, \,a>0
\end{equation}
%
The rate of convergence $a$ must be positive and $p$ defines the order of
convergence, most importantly linear convergence ($p=1$) and quadratic
convergence ($p=2$). An algorithm is generally desired to convergence
quadratically towards $\mathbf{x^*}$, however, linear convergence can be
acceptable in case the rate constant is low.

An optimisation algorithm is usually based on a model, that approximates the
objective function and allows for the prediction of the location of
$\mathbf{x^*}$. The methods used in this work belong to the group of quadratic
models and use a line search approach to locate the local minimiser. Focus will
therefore be put on this kind of approach. The idea of a line search relies on
a user-defined starting point $\mathbf{x}^{(1)}$ and is restricted to search
for a minimiser along coordinate directions. In a line search procedure the
$k$th iteration requires 
\begin{enumerate}
        \item to determine the search direction $\mathbf{s}^{(k)}$, 
        \item minimise $f\left(\mathbf{x}^{(k)}+\alpha\mathbf{s}^{(k)}\right)$ with respect to $\alpha$ and 
        \item to set the new iterate $\mathbf{x}^{(k+1)}=\mathbf{x}^{(k)}+\alpha^{(k)}\mathbf{s}^{(k)}$.
\end{enumerate}            
Step 1 depends on the chosen method, while step 2 is independent of the chosen
method and corresponds to the line search step. Step 2 is solved by sampling
$f(\mathbf{x})$ along the line $\mathbf{s}^{(k)}$ and, in practice, needs to be
terminated when a convergence criterion is met. This is because an exact
line-search, which corresponds to sampling
$f\left(\mathbf{x}^{(k)}+\alpha\mathbf{s}^{(k)}\right)$ until the true minimum
has been found, is not possible to be implemented with a finite amount of
steps.  Especially for points far from $\mathbf{x^*}$ it is sensible to choose
loose convergence criteria and tighten them around $\mathbf{x^*}$.

The line-search convergence criterion is usually not a user-defined value,
however, the termination criterion $\varepsilon$ of the optimisation procedure
needs to be supplied by the user. There are several possibilities of choosing a
convergence test with the most obvious approach being to test for convergence
in the minimum value $f^{(k)}-f^*\leq\varepsilon$ or the minimiser itself
$\left|x^{(k)}_i-x^*_i\right|\leq\varepsilon_i$. However, these criteria
require knowledge of exact minimiser or minimum value of the objective function
and it is easy to see, that this is paradoxical. A more useful criterion can be
based on the Euclidean norm of the gradient at the $k$th iterate
%
\begin{equation}
    \left|\left|\mathbf{g}^{(k)}\right|\right|\leq \varepsilon.
\end{equation}
%

\section{Quadratic Models}
\label{sec:QuadraticModels}

An optimisation method is said to be derived from a quadratic model if the
method approximates the objective function by a quadratic function around the
minimiser. A quadratic model has to be applied iteratively to a general
function to lead to convergence. Applied to a quadratic function it can be
shown that it should locate the minimiser in a finite amount of steps. The use
of a quadratic model has several advantages and most successful methods are
based on it. The most obvious way to derive a quadratic model is probably by
using information from both the gradient and the second derivatives, which
gives rise to the Newton-Raphson method (or quasi-Newton-Raphson if second
derivatives are estimated). However, it is possible to build a quadratic method
without using second derivatives and one such example is the conjugate gradient
method. 

\subsection{Newton-like Methods}
\label{sec:NewtonlikeMethods}

As mentioned above, a quadratic model can be derived by including information
from the second derivatives, which in the case of Newton-like methods is
achieved by truncating a Taylor expansion of the objective function around the
iterate minimiser $\mathbf{x}^{(k)}$.
%
\begin{equation}
    f\left(\mathbf{x}^{(k)}+\bm{\delta}\right)\approx 
        q^{(k)}(\bm{\delta})=
        f^{(k)} + {\mathbf{g}^{(k)}}^\mathbf{T}\bm{\delta}
        + \frac{1}{2}\bm{\delta^T}\mathbf{G}^{(k)}\bm{\delta}\label{eqn:TaylorNewton}
\end{equation}
% 
Here, $\bm{\delta}=\mathbf{x}-\mathbf{x}^{(k)}$ and $q^{(k)}(\bm{\delta})$ is
the quadratic approximation of the objective function around $\mathbf{x^*}$.
The next step in the optimisation $\mathbf{x}^{(k+1)}$ is then chosen based on
$\bm{\delta}=\bm{\delta}^{(k)}$ which minimises $q^{(k)}(\bm{\delta})$. It can
be shown, that the derivative $\nabla q^{(k)}(\bm{\delta})$ can be
expressed as
%
\begin{align}
    \nabla q^{(k)}(\bm{\delta})=\mathbf{G}^{(k)}\bm{\delta}+\mathbf{g}^{(k)},\label{eqn:QuadraticAprroximationDerivative}
\end{align}
%
and is said to be $0$ at $\bm{\delta}=\bm{\delta}^{(k)}$. The last conditions
results in $n\times n$ linear equations that can be solved programmatically,
and the result can be used to construct the next iterate
$\mathbf{x}^{(k+1)}=\mathbf{x}^{(k)}+\bm{\delta}^{(k)}$.

In practice Newton's method is usually implemented in combination with a
line-search algorithm. Because it is not guaranteed that the iterates
$\left\{f^{(k)}\right\}$ decrease towards the minimum value it can be useful to
define the direction of search as
%
\begin{align}
    \mathbf{s}^{(k)}=-{\mathbf{G}^{(k)}}^{-1}\mathbf{g}^{(k)}.
\end{align}
%
If $\mathbf{G}$ is positive definite so is $\mathbf{G}^{-1}$ and $\mathbf{s}$ is
a descent direction.

Problems arise if $\mathbf{G}$ is not positive definite, which can happen if
the current iterator is far from the local minimiser. In that case it is still
possible to calculate a search direction and search along positive and negative
direction to find a lower $f^{(k)}$. This means, however, the approximate
quadratic function does not yield a minimum objective function values
minimising point. One possible solution proposed by Goldstein and
Price\autocite{Goldstein_effectivealgorithmminimization_1967} is to iterate in
a steepest descent direction $\mathbf{s}^{(k)}=-\mathbf{g}^{(k)}$ in case the
Hessian matrix is not positive definite. Nonetheless, this method is prone to
oscillatory behaviour that would be introduced into the optimisation iteration.

If the Hessian matrix is almost positive definite a feasible approach might be
to modify $\mathbf{G}^{(k)}$ to make it positive definite. A better search
direction can be defined by adding a small multiple $\nu$ of a unit matrix
$\mathbf{I}$.
\autocite{Levenberg_methodsolutioncertain_1944,Marquardt_AlgorithmLeastSquaresEstimation_1963,Goldfeld_MaximizationQuadraticHillClimbing_1966}
%
\begin{align}
\mathbf{s}^{(k)}=-\left(\mathbf{G}^{(k)}+\nu\mathbf{I}\right)^{-1}\mathbf{g}^{(k)}
\end{align}
%
In this approach the quadratic information is still used, but no oscillatory
behaviour is being introduced. Instead of modifying the Hessian matrix with
multiples of the unit matrix it can also be modified more generally with a
diagonal matrix $\mathbf{D}$, which is advantageous in case the Hessian matrix
is indefinite.
\autocite{Murray_Secondderivativemethods_1972,Hebden_algorithmminimizationusing_1973}

Finally, the problem can be solved by computing a negative curvature descent
direction by solving
%
\begin{align}
    {\mathbf{s}^{(k)}}^\mathbf{T}\mathbf{G}^{(k)}\mathbf{s}^{(k)}<0\\
    {\mathbf{s}^{(k)}}^\mathbf{T}\mathbf{g}^{(k)}\leq0
\end{align}
for $\mathbf{s}^{(k)}$.\autocite{Fiacco_NonlinearProgramming_1990}


\subsection{Quasi-Newton Methods}
\label{sec:QuasiNewtonMethods}

Especially for quantum chemical potentials like Hartree-Fock or density
functional theory potentials the evaluation of a full Hessian matrix can be
very computationally expensive. It can therefore be useful to just use an
approximation for the Hessian matrix. n the most simple case this results in a
finite difference Newton method where $\mathbf{G}^{(k)}$ is approximated in
terms of differences of the gradient vector with respect to each coordinate
direction $\mathbf{e_i}$.
%
\begin{align}
    {\bm{\Delta}\mathbf{g}}_\mathbf{i}^{(k)}=\frac{1}{h_i}\left[\mathbf{g}\left(\mathbf{x}^{(k)}+h_i\mathbf{e_i}\right)-\mathbf{g}^{(k)}\right]\\
    \mathbf{G}^{(k)}\approx \mathbf{\overline{G}}=
    \begin{pmatrix}
        \bm{\Delta}\mathbf{g}_\mathbf{1}^{(k)} & \bm{\Delta}\mathbf{g}_\mathbf{2}^{(k)} & \cdots & \bm{\Delta}\mathbf{g}_\mathbf{n}^{(k)}
    \end{pmatrix}
\end{align}
%
The approximated matrix needs to be symmetrised by calculating
$\frac{1}{2}\left(\mathbf{\overline{G}}+\mathbf{\overline{G}^T}\right)$. However, this
approach has some disadvantages, e.g. the symmetrised matrix is not guaranteed
to be positive definite and for the calculation of $\mathbf{\overline{G}}$ the
gradient has to be evaluated $n$ times making this approximation potentially as
expensive as the exact Hessian calculation.

The class of quasi-Newton methods tries to avoid these problems by
approximating ${\mathbf{G}^{(k)}}^{-1}$ with a symmetric positive definite
matrix $\mathbf{H}^{(k)}$, which can then be updated each iteration. The $k$th
iteration of a quasi-Newton optimisation requires to
%
\begin{enumerate}
    \item determine the search direction $\mathbf{s}^{(k)}=-\mathbf{H}^{(k)}\mathbf{g}^{(k)}$,
    \item minimise $f\left(\mathbf{x}^{(k)}+\alpha\mathbf{s}^{(k)}\right)$ in a line-search procedure,
    \item set the new iterate $\mathbf{x}^{(k+1)}=\mathbf{x}^{(k)}+\alpha^{(k)}\mathbf{s}^{(k)}$ and
    \item update $\mathbf{H}^{(k)}$.
\end{enumerate}
%
The initial choice of $\mathbf{H}^{(1)}$ is not important as long as the matrix
is symmetric positive definite. Simply choosing a unit matrix is sufficient,
which turns the first step of the optimisation into a steepest descent step as
$\mathbf{s}^{(k)}=-\mathbf{g}^{(k)}$.

The method is practically identical with a line-search Newton-like method, with
the difference being the representation of the matrix of second derivatives.
The step in the procedure that defines and updates $\mathbf{H}^{(k)}$ is
therefore very important for quasi-Newton methods. The goal is that updating
$\mathbf{H}^{(k)}$ in each iteration to $\mathbf{H}^{(k+1)}$ results in a good
approximation for ${\mathbf{G}^{(k)}}^{-1}$. Using
equation~\eqref{eqn:QuadraticAprroximationDerivative} and choosing
$\mathbf{x}=\mathbf{x}^{(k+1)}$ such that
$\bm{\delta}^{(k)}=\mathbf{x}^{(k+1)}-\mathbf{x}^{(k)}$ it is easy to see that
in the quadratic approximation the difference between the gradient vectors
$\bm{\gamma}^{(k)}=\mathbf{g}^{(k+1)}-\mathbf{g}^{(k)}$ is mapped to the
distance vector between the points by the Hessian matrix.
%
\begin{align}
    \bm{\gamma}^{(k)}=\mathbf{G}^{(k)}\bm{\delta}^{(k)}
\end{align}
%
However, $\mathbf{x}^{(k)}$ is only known after the line-search completed,
which means that $\mathbf{H}^{(k)}$ (the inverse of $\mathbf{G}^{(k)}$) does
not map them properly. Yet, this relation can be used to improve the
approximated inverse Hessian matrix for the next step $\mathbf{H}^{(k+1)}$.
%
\begin{align}
    \mathbf{H}^{(k+1)}\bm{\gamma}^{(k)}=\bm{\delta}^{(k)}\label{eqn:QuasiNewtonCondition}
\end{align}
%
This is the so-called quasi-Newton condition and the differences in different
quasi-Newton methods lie within the way this condition is fulfilled
computationally.

One way to generate $\mathbf{H}^{(k+1)}$ is to update $\mathbf{H}^{(k)}$ by
adding a symmetric rank one matrix $\mathbf{E}^{(k)}=a\mathbf{uu^T}$.
%
\begin{align}
    \mathbf{H}^{(k+1)}=\mathbf{H}^{(k)}+a\mathbf{uu^T}
\end{align}
%
Using equation~\eqref{eqn:QuasiNewtonCondition} it can be seen that
$\mathbf{u}$ is proportional to
$\bm{\delta}^{(k)}-\mathbf{H}^{(k)}\bm{\gamma}^{(k)}$ with a proportionality
constant of $a\mathbf{u^T}\bm{\gamma}^{(k)}$. Since the proportionality can
just be chosen to be one by changing $a$, it follows that
$\mathbf{u}=\bm{\delta}^{(k)}-\mathbf{H}^{(k)}\bm{\gamma}^{(k)}$ and hence the
formula for updating $\mathbf{H}^{(k)}$ can be expressed as
%
\begin{align}
    \mathbf{H}^{(k+1)}=\mathbf{H}+\frac{(\bm{\delta}-\mathbf{H}\bm{\gamma})(\bm{\delta}-\mathbf{H}\bm{\gamma})^\mathbf{T}}{(\bm{\delta}-\mathbf{H}\bm{\gamma})^\mathbf{T}\bm{\gamma}}.\label{eqn:RankOneQuasiNewton}
\end{align}
%
The superscript $(k)$ has been omitted on the right sight to improve
readability and this notation will be adopted for following update formulae as well.
Originally, this formula was proposed by multiple people independently.
\autocite{Broyden_QuasiNewtonMethodstheir_1967,Davidon_VarianceAlgorithmMinimization_1968,Fiacco_NonlinearProgramming_1990}
It is natural to assume that this formula could be improved by introducing a
second independent change to $\mathbf{H}^{(k)}$.
%
\begin{align}
    \mathbf{H}^{(k+1)}=\mathbf{H}^{(k)}+a\mathbf{uu^T}+b\mathbf{vv^T}
\end{align}
%
Unfortunately, the expressions for $\mathbf{u}$ and $\mathbf{v}$ can not be
established as easily as before. However, $\mathbf{u}=\bm{\delta}^{(k)}$ and
$\mathbf{v}=\mathbf{H}^{(k)}\bm{\gamma}^{(k)}$ have shown to be sensible
choices and give rise to the \ac{DFP}
formula.\autocite{Davidon_VariableMetricMethod_1991,Fletcher_RapidlyConvergentDescent_1963}
%
\begin{align}
    \mathbf{H}_\mathbf{DFP}^{(k+1)}=\mathbf{H}+\frac{\bm{\delta\delta}^\mathbf{T}}{\bm{\delta}^\mathbf{T}\bm{\gamma}}-\frac{\mathbf{H}\bm{\gamma\gamma}^\mathbf{T}\mathbf{H}}{\bm{\gamma}^\mathbf{T}\mathbf{H}\bm{\gamma}}\label{eqn:DFPQuasiNewton}
\end{align}
%
The probably most successful quasi-Newton method is based on the \ac{BFGS}
formula,\autocite{Broyden_ConvergenceClassDoublerank_1970,Broyden_ConvergenceClassDoublerank_1970a,Fletcher_newapproachvariable_1970,Goldfarb_familyvariablemetricmethods_1970,Shanno_ConditioningquasiNewtonmethods_1970}
which is closely related to the \ac{DFP} formula.
%
\begin{align}
    \mathbf{H}_\mathbf{BFGS}^{(k+1)}=\mathbf{H}
    +\left(1+\frac{\bm{\gamma}^\mathbf{T}\mathbf{H}\bm{\gamma}}{\bm{\delta}^\mathbf{T}\bm{\gamma}}\right)
    \frac{\bm{\delta\delta}^\mathbf{T}}{\bm{\delta}^\mathbf{T}\bm{\gamma}}
    -\left(\frac{\bm{\delta\gamma}^\mathbf{T}\mathbf{H}+\mathbf{H}\bm{\gamma\delta}^\mathbf{T}}{\bm{\delta}^\mathbf{T}\bm{\gamma}}\right)\label{eqn:BFGSQuasiNewton}
\end{align}
%
The relation can be illustrated by denoting $\mathbf{H}^{-1}$ as $\mathbf{B}$
and substitute in in equation~\eqref{eqn:BFGSQuasiNewton}.
%
\begin{align}
    \mathbf{B}_\mathbf{BFGS}^{(k+1)}=\mathbf{B}+\frac{\bm{\gamma\gamma}^\mathbf{T}}{\bm{\gamma}^\mathbf{T}\bm{\delta}}
    -\frac{\mathbf{B}\bm{\delta\delta}^\mathbf{T}\mathbf{B}}{\bm{\delta}^\mathbf{T}\mathbf{H}\bm{\delta}}
\end{align}
%
The similarity to equation~\eqref{eqn:DFPQuasiNewton} is obvious. Because both
formulae are related by exchanging $\bm{\gamma}\leftrightarrow\bm{\delta}$ and
$\mathbf{B}\leftrightarrow\mathbf{H}$ they are called dual or complementary.
%maybe talk about BFGS/DFP being invariant under affine transformation?

\subsection{Conjugate Gradient Methods}
\label{sec:ConjugateGradientMethods}

The origin of the Newton-like and quasi-Newton methods being the quadratic
model is conceptually obvious. There are however methods that belong into this
group, but don't rely on calculating or approximating a matrix of second
derivatives. One of those methods is the conjugate gradient method.  As the
name suggests, they take advantage of the concept of conjugacy of the search
vectors $\mathbf{s}^{(1)},\mathbf{s}^{(2)},\dots,\mathbf{s}^{(n)}$ the Hessian
matrix $\mathbf{G}$, i.e.
%
\begin{align}
    {\mathbf{s}^{(i)}}^\mathbf{T}\mathbf{G}\mathbf{s}^{(j)}=0,\,\forall i\neq j.\label{eqn:ConjugacyCondition}
\end{align}
%
It should be noted that quadratic termination is only guaranteed for exact line
searches. The conjugate gradient method tries to combine the conjugacy property
with the steepest descent method, therefore the first step is equal to
%
\begin{align}
    \mathbf{s}^{(1)}=-\mathbf{g}^{(k)}
\end{align}
%
and for successive iterations
%
\begin{align}
    \mathbf{s}^{(k+1)}=\text{component of }-\mathbf{g}^{(k+1)}\text{conjugate to }\mathbf{s}^{(1)},\mathbf{s}^{(2)},\dots,\mathbf{s}^{(n)}.
\end{align}
%
From the conjugacy condition \eqref{eqn:ConjugacyCondition} it follows that
$\mathbf{s}^{(k+1)}$ can be calculated from a Gram-Schmidt orthonormalisation.
%
\begin{align}
    \mathbf{s}^{(k+1)}=-\mathbf{g}^{(k+1)}+\beta^{(k)}\mathbf{s}^{(k)}\\
    \beta^{(k)}=\frac{{\mathbf{g}^{(k+1)}}^\mathbf{T}\mathbf{g}^{(k+1)}}{{\mathbf{g}^{(k)}}^\mathbf{T}\mathbf{g}^{(k)}}\label{eqn:FletcherReevesBeta}
\end{align}
%
This is also known as the \ac{FR}
method.\autocite{Fletcher_Functionminimizationconjugate_1964} One advantage of
the \ac{FR} method over the quasi-Newton methods is that it doesn't need any
matrix calculation, which can be seen in
equation~\eqref{eqn:FletcherReevesBeta}. However, the procedure is not
guaranteed to terminate for non-quadratic functions. There are several ways
that try to solve this disadvantage, one of which is a simple reset of the
search direction to the steepest descent direction. If the iterates converge
towards a region that is approximated well by a quadratic function, then the
reset of the search direction may improve the overall convergence of the
method.

Another solution to the aforementioned problem is to use a different expression
for $\beta^{(k)}$. One possible modification is
%
\begin{align}
    \beta^{(k)}=-\frac{{\mathbf{g}^{(k+1)}}^\mathbf{T}\mathbf{g}^{(k+1)}}{{\mathbf{g}^{(k)}}^\mathbf{T}\mathbf{s}^{(k)}}\label{eqn:DescentBeta},
\end{align}
%
which results in a stronger descent property
${\mathbf{s}^{(k)}}^\mathbf{T}\mathbf{g}^{(k)} < 0$. Another notable mention is
the formula by Polak and
Ribiere\autocite{Polak_ComputationalMethodsOptimization_1971} shown in
equation~\eqref{eqn:PolakRibiereBeta}.
%
\begin{align}
    \beta^{(k)}=\frac{\left(\mathbf{g}^{(k+1)}-\mathbf{g}^{(k)}\right)^\mathbf{T}\mathbf{g}^{(k+1)}}{{\mathbf{g}^{(k)}}^\mathbf{T}\mathbf{g}^{(k)}}\label{eqn:PolakRibiereBeta}
\end{align}

\section{Practical Implementation for Two-Body Potentials}
\label{sec:PracticalImplementationForPotentialsDependingOnPairDistances}

In the special case of optimising the geometry of a collection of objects in
three-dimensional space that interact via a given potential some modifications
have to be made to use the previously described methods. In the following
paragraphs the mathematical background for the implementation of potentials
that only depend on the distance between two objects like \ac{LJ} and \ac{eLJ}
for the program \textsc{Spheres} (chapter~\ref{sec:theprogramspheres}) is
explained. The physical objects in this case are called spheres and the
optimisation procedure tries to locate the minimiser corresponding to the
lowest total energy of the system.

Let $\mathbf{x}_i$ be the Cartesian coordinates of sphere $i$ and the
collection of all the coordinates of all $N$ spheres in the system shall be
denoted $\mathbf{X}$.
%
\begin{align}
    \begin{aligned}
        \mathbf{X}=
        \begin{pmatrix}
            \mathbf{x}_1 & \mathbf{x}_2 & \cdots & \mathbf{x}_N
        \end{pmatrix}&=
        \begin{pmatrix}
            x_1 & x_4 & \cdots & x_{3N-2}\\
            x_2 & x_5 & \cdots & x_{3N-1}\\
            x_3 & x_6 & \cdots & x_{3N}
        \end{pmatrix}\\
            \mathbf{x}_i &= 
            \begin{pmatrix}
                x_{3i-2}\\
                x_{3i-1}\\
                x_{3i}
            \end{pmatrix}
    \end{aligned}
\end{align}%
%
The distance between two spheres $i$ and $j$ is now given by the norm of the
distance vector $\mathbf{r}_{ij}$. %
%
\begin{align}
    \mathbf{r}_{ij}=\mathbf{x}_i-\mathbf{x}_j=
    \begin{pmatrix}
        x_{3i-2} - x_{3j-2}\\
        x_{3i-1} - x_{3j-1}\\
        x_{3i} - x_{3j}
    \end{pmatrix}\label{eq:distancevector}\\
    |\mathbf{r}_{ij}|=r_{ij}=\sqrt{(x_{3i-2} - x_{3j-2})^2 + (x_{3i-1} - x_{3j-1})^2 + (x_{3i} - x_{3j})^2}\label{eq:distance}
\end{align}
%
The energy of the system is a function of all sphere coordinates $\mathbf{X}$,
but it can be rewritten in terms of an energy function $\varepsilon(r_{ij})$ that only depends on the
distance between two spheres. %
%
\begin{align}
    E(\mathbf{X})=\sum_{j>i}^N\varepsilon(r_{ij})
\end{align}%
%
The gradient of the system is a vector of $3N$ scalars, where each component
refers to the gradient with respect to each coordinate $x$. The derivative with
respect to the $m$th coordinate $x_m$ can be expressed as in
equation~\eqref{eqn:gradientcomponent} by using the chain rule for derivatives.
%
\begin{align}
    \pdv{E(\mathbf{X})}{x_m}=\sum_{j>i}^N\pdv{\varepsilon(r_{ij})}{r_{ij}}\ \pdv{r_{ij}}{\mathbf{r}_{ij}}\ \pdv{\mathbf{r}_{ij}}{x_m}\label{eqn:gradientcomponent}
\end{align}%
%
It is clear, that the terms that contain vectors are separated from the energy
function. This means that the energy term can be exchanged easily without
having to change all parts of the routine. The first term represents the
derivative of the energy function with respect to the distance between two
spheres. The second term can be rewritten in terms of the normalised form of
the distance vector $\mathbf{r_{ij}}$, which follows directly from
equations~\eqref{eq:distancevector} and \eqref{eq:distance}.
%
\begin{align}
    \pdv{r_{ij}}{\mathbf{r}_{ij}}=\frac{\mathbf{r}_{ij}}{r_{ij}}
\end{align}%
%
The last term is responsible for the right sign of the gradient component and
is best explained by giving an example. Firstly, if $x_m$ is neither in sphere
$i$ nor in $j$ its result is a zero vector making the whole expression vanish.
Let's assume $m=3i+1$, then the last expression becomes:
%
\begin{align}
    \pdv{\mathbf{r}_{ij}}{x_{3i-1}}=
    \begin{pmatrix}
        0\\1\\0
    \end{pmatrix}.
\end{align}%
%
For this example the inner product of this vector with the normalised distance
vector $\mathbf{r}_{ij}$ is $\frac{1}{r_{ij}}(x_{3i-1} - x_{3j-1})$. Therefore,
the last term ensures that the $m$th component of the gradient vector only
collects contributions from interactions between spheres that contain the
coordinate $x_m$. If $m$ was a coordinate present in sphere $j$ the last term
swaps the sign of the gradient. This is a result of the fact, that the gradient
at sphere $j$ should be opposite of the gradient at sphere $i$. The final
gradient is given by calculating all partial derivatives with respect to $x_m$.
%
\begin{align}
    \nabla E(\mathbf{X})=
    \begin{pmatrix}
        \pdv*{E(\mathbf{X})}{x_1}\\
        \pdv*{E(\mathbf{X})}{x_2}\\
        \vdots\\
        \pdv*{E(\mathbf{X})}{x_{3N}}
    \end{pmatrix}
\end{align}
%
The separation of the vector and scalar components allows for easy exchange of
the energy functions as the calculations that take care of the direction of the
gradient can be completely separated out. 

The same procedure can be applied to the second derivative to calculate a
Hessian matrix. Again, the important part is to separate the scalar energy
function from vector parts. This leads to the following equations.
%
\begin{align}
    \pdv{E(\mathbf{X})}{x_m}{x_n}=\sum_{j>i}^N\left[  
    \begin{array}{ll}
    \pdv[2]{\varepsilon(r_{ij})}{{r_{ij}}}\  
    \pdv{r_{ij}}{\mathbf{r}_{ij}}\ 
    \pdv{\mathbf{r}_{ij}}{x_m}\ 
    \pdv{r_{ij}}{\mathbf{r}_{ij}}\ 
    \pdv{\mathbf{r}_{ij}}{x_n} \\ 
    +
    \pdv{\varepsilon(r_{ij})}{r_{ij}}\
        \pdv[2]{r_{ij}}{{\mathbf{r}_{ij}}}\ 
    \pdv{\mathbf{r}_{ij}}{x_m}\ 
    \pdv{\mathbf{r}_{ij}}{x_n} 
    \end{array}
    \right]\\
    \begin{aligned}
    \nabla^2 E(\mathbf{X})=\\
    \begin{pmatrix}
        \pdv*[2]{E(\mathbf{X})}{{x_1}} & \pdv*{E(\mathbf{X})}{x_1}{x_2} & \ldots & \pdv*{E(\mathbf{X})}{x_1}{x_{3N}}\\
        \pdv*{E(\mathbf{X})}{x_2}{x_1} & \pdv*[2]{E(\mathbf{X})}{{x_2}} & \ldots & \pdv*{E(\mathbf{X})}{x_2}{x_{3N}}\\
        \vdots & \vdots & \ddots & \vdots\\
        \pdv*{E(\mathbf{X})}{x_{3N}}{x_1} & \pdv*{E(\mathbf{X})}{x_{3N}}{x_2} & \ldots & \pdv*[2]{E(\mathbf{X})}{{x_{3N}}}
    \end{pmatrix}
    \end{aligned}
\end{align}



\chapter{Energy Landscapes}
\label{sec:energylandscapes}

\section{Lennard-Jones}
\label{sec:LennardJones}

\section{Sticky-Hard-Sphere}
\label{sec:SHS}
