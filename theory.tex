%theory

\chapter{Theory}
\label{sec:theory}

The following chapters should provide an overview of the theoretical backround
neccesarry to carry out the presented calculations. Most of it is based on
standard literature from \textsc{Jensen}\autocite{jensen_introduction_2006}, \textsc{Szabo} and
\textsc{Ostlund}\autocite{szabo_modern_1996} and \textsc{Holthausen} and
\textsc{Koch}\autocite{koch_chemists_2001}.

\section{The fundamentals of Quantum Mechanics}
\label{sec:thefundamentalsofQM}

\subsection{The Schr\"odinger Equation}

The discovery of the wave-particle duality at the beginning of the 20th
centuary lead to a complete reformulation of the physical laws governing the
smallest of particles. \textsc{Erwin Schr\"odinger} established a framework
based on \textsc{Hamilton}ian mechanics that set the wavefunction $\Psi$ at the
center of attention. It contains all information about the system and its
evolution in time. For ground-state calculations it is usually sufficient to
look at solutions to the time-independent \textsc{Schr\"odinger} equation.

\begin{equation}
	\mathbf{H}\Psi=E\Psi
\end{equation}

In this eigenvalue equation the \textsc{Hamilton} operator $\mathbf{H}$ acts on
the wavefunction which results in a solution for the total energy of the system
$E$. For a system of $N$ electrons and $M$ nuclei the non-relativistic
\textsc{Hamilton} operator takes the form

\begin{align}
	\mathbf{H}&=\mathbf{T} + \mathbf{V} \\
	\mathbf{T}&=\sum_{i=1}^N=-\sum_{i=1}^N\frac{1}{2m_i}\nabla_i^2 \\
	\mathbf{V}&=\sum_{i}^N\mathbf{V}_{ij} \\
	-\frac{1}{2}\sum_{i=1}^N\nabla_i^2-\frac{1}{2}\sum
\end{align}
