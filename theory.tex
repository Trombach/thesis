%theory

\chapter{Theory}
\label{sec:theory}


\section{An Introduction to Quantum Chemistry}
\label{sec:IntroductiontoQM}

Parts of this thesis require knowledge of standard methods used in quantum
chemistry. The following chapters introduce the fundamentals of quantum
mechanics and their application to chemistry related questions. Most of it is
based on standard literature by Jensen\autocite{jensen_introduction_2006},
Szabo and Ostlund\autocite{szabo_modern_1996} and Holthausen and
Koch\autocite{koch_chemists_2001}.

\subsection{The Schr\"odinger Equation}

The beginning of the 20th centuary was a very important time for modern
theoretical sciences. Discoveries like Planck's energy quantisation based on
black body radiation or the discovery of the wave particle dualism by de
Broglie lead to a complete reformulation of the physical laws governing the
smallest of particles. Erwin Schr\"odinger established a framework based on
Hamiltonian mechanics that set the wave function $\Psi$ at the center of
attention. It contains all information about the system and its evolution in
time. For ground-state calculations it is usually sufficient to look at
solutions to the time-independent Schr\"odinger equation.

\begin{equation}
	\mathbf{H}\Psi=E\Psi\label{eqn:SchrodingerEquation}
\end{equation}

In this eigenvalue equation the Hamilton operator $\mathbf{H}$ acts on the wave
function which results in a solution for the total energyi $E$ of the system. 

The wave function is a function of all spatial $\mathbf{r}$ and spin
coordinates $\omega$ of all the particles in the system and will be denoted
$\mathbf{x}$ on the following pages.  

\begin{align}
	\mathbf{x}=\{\mathbf{r},\omega\}
\end{align}

The square of the wave function is usually interpreted as a probability density
which is used to normalise it such that the probability of finding and electron
anywhere in space is one.

\begin{align}
	\int |\Psi(\mathbf{x})|^2\dif \mathbf{x}_1\dif \mathbf{x}_2\ldots\dif \mathbf{x}_Ni=\braket{\Psi(\mathbf{x})|\Psi(\mathbf{x})}=1
\end{align}

The information contained in the wave function can be accessed by an operator
$\mathbf{O}$ acting on the wave function and forming a eigenvalue equation. In
equation~\eqref{eqn:SchrodingerEquation} this operator is the Hamilton
operator, but it could be any operator connected to a physical observable.
Generally the eigenvalue of an arbitrary operator $\braket{\mathbf{O}}$ is
defined as

\begin{equation}
	\braket{\mathbf{O}}=\frac{\braket{\Psi(\mathbf{x})|\mathbf{O}|\Psi(\mathbf{x})}}{\braket{\Psi(\mathbf{x})|\Psi(\mathbf{x})}}
\end{equation}

and for normalised wave functions this becomes

\begin{align}
	\braket{\mathbf{O}}=\braket{\Psi(\mathbf{x})|\mathbf{O}|\Psi(\mathbf{x})}.
\end{align}

For a system of $N$ electrons and $M$ nuclei the non-relativistic Hamilton
operator in atomic units has the form

\begin{align}
\begin{aligned}
    \mathbf{H}&=\mathbf{T}_\text{e} + \mathbf{V}_\text{ee} + \mathbf{T}_\text{n} + \mathbf{V}_\text{nn} + \mathbf{V}_\text{en} \\
    &=-\sum_{i=1}^N\frac{1}{2}\nabla_i^2
    + \sum_{i=1}^N\sum_{j>i}^NV_{ij}
    - \sum_{A=1}^M\frac{1}{2}\nabla_A^2
    + \sum_{A=1}^M\sum_{B>A}^MV_{AB}
    + \sum_{i=1}^N\sum_{A=1}^MV_{iA}.\label{eqn:hamiltonoperator}
\end{aligned}
\end{align}

The operator contains the kinetic energy of the electrons
$\mathbf{T}_\text{e}$, the electron repulsion $\mathbf{V}_\text{ee}$, the
kinetic energy of the nuclei $\mathbf{T}_\text{n}$, the nucleon repulsion
$\mathbf{V}_\text{nn}$ and the electron-nucleon attraction
$\mathbf{V}_\text{en}$.

\subsection{The Born-Oppenheimer Approximation}

For quantum chemical applications the coupling of the movement of the electrons
and nuclei is usually neglected. This is possible because the atomic mass $m_A$
is so much greater than the electronic mass $m_i$, resulting in much lower
velocities for nuclei compared to electrons. Therefore, the electrons can be
considred to be moving in a static field of nuclei, meaning the nuclear kinetic
term $\mathbf{T}_\text{n}$ can be neglected and the nuclear repulsion term
$\mathbf{V}_\text{nn}$ becomes a constant. The resulting electronic Hamilton
operator $\mathbf{H}_\text{e}$ describes electrons moving in a field of
positive point charges.

\begin{align}
    \mathbf{H}_\text{e}=\mathbf{T}_\text{e} + \mathbf{V}_\text{ee} + \mathbf{V}_\text{en}
\end{align}

Solving the Schr\"dinger equation for this operator yields the electronic wave
function $\Psi_\text{e}$. It depends explicitly on the electronic coordinates,
but only parametrically on the nuclear coordinates, therefore spanning a
potential energy surface upon which the nuclei move. To get the total energy of
the system the constant repulsion between the nuclei has to be added to the
electronic operator.

\begin{align}
    \mathbf{H}=\mathbf{H}_\text{e}+V_\text{nn}
\end{align}

\subsection{The Hartree-Fock Approximation}

An exact solution to the Schr\"odinger equation in the boundary condition of
the Born-Oppenheimer approximation ifor systems containing more than one
electron is not possible. The solution has to be approximated by appropriate
methods; one possible solution are the Hartree-Fock equations. This set of
equations determines the energy of a wave function expressed as a Slater
determinant $\Phi_\text{SD}$.

\begin{equation}
     \Phi_{SD}=\frac{1}{\sqrt{N!}}
     \begin{vmatrix}
         \phi_1(\mathbf{x}_1) & \phi_2(\mathbf{x}_1) & \cdots & \phi_N(\mathbf{x}_1)\\
         \phi_1(\mathbf{x}_2) & \phi_2 (\mathbf{x}_2) & \cdots & \phi_N(\mathbf{x}_2)\\
         \vdots & \vdots & \ddots & \vdots\\
         \phi_1(\mathbf{x}_N) & \phi_2(\mathbf{x}_N) & \cdots & \phi_N(\mathbf{x}_N)
     \end{vmatrix}
     \label{eqn:SlaterDet}
\end{equation}

Here, the $\phi_i$ denote one electron spin orbitals and
$\mathbf{x}_i=\{\mathbf{r}_i,\omega_i\}$ are spatial ($\mathbf{r}_i$) and spin
($\omega_i$) coordinates of the electrons. A Slater determinant obeys the Pauli
exclusion principle, which requires the electronic wave function to be
anti-symmetric with respect to interchanging the coordinates of two electrons.
