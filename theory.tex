%theory

\chapter{Theory}
\label{sec:theory}


\section{An Introduction to Quantum Chemistry}
\label{sec:IntroductiontoQM}

The following chapters should provide an overview of the theoretical backround
neccesarry to carry out the presented calculations. Most of it is based on
standard literature by Jensen\autocite{jensen_introduction_2006},
Szabo and Ostlund\autocite{szabo_modern_1996} and
Holthausen and Koch\autocite{koch_chemists_2001}.

\subsection{The Schr\"odinger Equation}

The discovery of the wave-particle duality at the beginning of the 20th
centuary lead to a complete reformulation of the physical laws governing the
smallest of particles. Erwin Schr\"odinger established a framework
based on Hamiltonian mechanics that set the wave function $\Psi$ at the
center of attention. It contains all information about the system and its
evolution in time. For ground-state calculations it is usually sufficient to
look at solutions to the time-independent Schr\"odinger equation.

\begin{equation}
	\mathbf{H}\Psi=E\Psi
\end{equation}

In this eigenvalue equation the Hamilton operator $\mathbf{H}$ acts on
the wave function which results in a solution for the total energy of the system
$E$. For a system of $N$ electrons and $M$ nuclei the non-relativistic
Hamilton operator has the form

\begin{align}
\begin{aligned}
    \mathbf{H}&=\mathbf{T}_\text{elec} + \mathbf{V}_\text{elec} + \mathbf{T}_\text{nuc} + \mathbf{V}_\text{nuc} + \mathbf{V}_\text{elec,nuc} \\
    \mathbf{T}_\text{elec}&=\sum_{i=1}^N\mathbf{T}_i=-\sum_{i=1}^N\frac{1}{2m_i}\nabla_i^2 \\
    \mathbf{V}_\text{elec}&=\sum_{i=1}^N\sum_{j>i}^N\mathbf{V}_{ij} \\
    \mathbf{T}_\text{nuc}&=\sum_{A=1}^M\mathbf{T}_A=-\sum_{A=1}^M\frac{1}{2m_A}\nabla_i^2 \\
    \mathbf{V}_\text{nuc}&=\sum_{A=1}^M\sum_{B>A}^M\mathbf{V}_{AB} \\
    \mathbf{V}_\text{elec,nuc}&=\sum_{i=1}^N\sum_{A=1}^M\mathbf{V}_{iA}.\label{eqn:hamiltonoperator}
\end{aligned}
\end{align}

The indices "elec" and "nuc" refer to whether the operator acts upon electrons
or nuclei, respectively. The Hamilton operator contains all kinetic
$\mathbf{T}$ and potential energy operators $\mathbf{V}$ to completely describe
the atomic system.

\subsection{The Born-Oppenheimer Approximation}

For quantum chemical applications the coupling of the movement of the electrons
and nuclei is usually neglected. This is possible because the atomic mass $m_A$
is so much greater than the electronic mass $m_i$, resulting in much lower
velocities for nuclei compared to electrons. Therefore, the electrons can be
considred to be moving in a static field of nuclei, meaning the nuclear kinetic
term $\mathbf{T}_\text{nuc}$ can be neglected and the nuclear repulsion term
$\mathbf{V}_\text{nuc}$ becomes a constant. The resulting electronic Hamilton
operator $\mathbf{H}_\text{elec}$ describes electrons moving in a field of
positive point charges.

\begin{align}
    \mathbf{H}_\text{elec}=\mathbf{T}_\text{elec} + \mathbf{V}_\text{elec} + \mathbf{V}_\text{elec,nuc}
\end{align}

Solving the Schr\"dinger equation for this operator yields the electronic wave
function $\Psi_\text{elec}$. It depends explicitly on the electronic
coordinates, but only parametrically on the nuclear coordinates, therefore
spanning a potential energy surface upon which the nuclei move.

\subsection{The Hartree-Fock Approximation}

An exact solution to the Schr\"odinger equation in the boundary condition of
the Born-Oppenheimer approximation ifor systems containing more than one
electron is not possible. The solution has to be approximated by appropriate
methods; one possible solution are the Hartree-Fock equations. This set of
equations determines the energy of a wave function expressed as a Slater
determinant $\Phi_\text{SD}$.

\begin{equation}
     \Phi_{SD}=\frac{1}{\sqrt{N!}}
     \begin{vmatrix}
         \phi_1(\mathbf{x}_1) & \phi_2(\mathbf{x}_1) & \cdots & \phi_N(\mathbf{x}_1)\\
         \phi_1(\mathbf{x}_2) & \phi_2 (\mathbf{x}_2) & \cdots & \phi_N(\mathbf{x}_2)\\
         \vdots & \vdots & \ddots & \vdots\\
         \phi_1(\mathbf{x}_N) & \phi_2(\mathbf{x}_N) & \cdots & \phi_N(\mathbf{x}_N)
     \end{vmatrix}
     \label{eqn:SlaterDet}
\end{equation}

Here, $\phi_i$ denote one electron spin orbitals and $\mathbf{x}_i$ are
electron coordinates. A Slater determinant obeys the Pauli exclusion principle,
which requires the electronic wave function to be anti-symmetric with respect
to interchanging the coordinates of two electrons.
