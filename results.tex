%results

\part{Results}
\label{sec:results}

\chapter{Golden Dual Fullerenes}
\label{sec:goldendualfullerenes}

\chapter{From Sticky-Hard-Sphere to Lennard-Jones-Type clusters}
\label{sec:fromstickyhardspheretoLJtypeclusters}

\chapter{The Gregory-Newton Clusters}
\label{sec:thegregorynewtonclusters}

\section{The Gregory-Newton Problem for Soft Potentials}
\label{sec:thegregorynewtonproblemforsoftpotentials}

The question of the Newton number in three dimensions has been resolved almost
70 years ago\autocite{schutte_problem_1952}. The proof is valid for hard-sphere
short-range potentials, but little is known about the behaviour of such
clusters under long-range potentials such as the Kratzer
potential\autocite{kratzer_ultraroten_1920}. We used the optimisation
procedures explained in chapter~\ref{sec:theprogramspheres} to minimise the
energy of a starting structure consisting of 13 spheres surrounding a center
sphere with a fixed distance of one. Generating such a starting structure where
all surrounding spheres are evenly spaced is impossible as there exists no
triangulation of a sphere with 13 vertices, where every vertex has degree five
or six\autocite{schwerdtfeger_topology_2015}. To generate an approximate
distribution we used a Fibonacci sphere
algorithm\autocite{gonzalez_measurement_2010,keinert_spherical_2015}.

\section{The Smallest Gregory-Newton Clusters}
\label{sec:themsmallestgregorynewtonclusters}

\section{Adding a 14th Sphere}
\label{sec:addinga14thsphere}
