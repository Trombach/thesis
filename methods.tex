%methods

\part{Methods}
\label{sec:methods}

\chapter{Quantum Chemical Programs}
\label{sec:quantumchemicalprograms}

\chapter{The Program \textsc{Spheres}}
\label{sec:theprogramspheres}

Chapters~\ref{sec:fromstickyhardspheretoLJtypeclusters} and
\ref{sec:thegregorynewtonclusters} required methods to optimise and analyse
arbitrary cluster structures based on a potential depending only on the
distance between two spheres. For this purpose the program package
\textsc{Spheres} was developed in \Cpp. The following chapters contain
information about the most important classes and functions of the program.

\section{General Structure}
\label{sec:generalstructure}

Most of the functions and classes of the program are contained in header files,
that can be included in any code file. This allows the package to be quite
flexible while maintaining readability of the code.

The most basic header file is \verb|geometry.h| which contains the classes
responsible for holding triplets of coordinates (\verb|coord3d|) and to
perform basic 3x3 matrix algebra (\verb|matrix3d|).

\section{The \texttt{structure} Class}
\label{sec:thestructureclass}

The \verb|structure| class provides methods to store coordinates of cluster
geometries and holds functions to calculate properties like energy and moment
of inertia and store them together with the coordinates in the same object. The
class relies on the \verb|coord3d| class for the storage of the coordinates of
the individual spheres.  When an object of this class is created by providing a
set of cartesian coordinates the center of the structure will be moved to the
center of mass of the system and the structure will be rotated such that the
main axis align with the principal axis of the structure. Additionally, all
inter-particle distances will be calculated and stored in the object so that
they are readily available for analysis.

For graph theoretical analysis the class provides functions to calculate
adjacency matrices and graphs based on the \textit{boost graph
library}\autocite{_boost_2002}. These methods take a threshold number that
defines the distance between two spheres at which they are considered
connected.

\section{The \texttt{pairPotential} Class}
\label{sec:thepairpotentialclass}



\section{Optimisation of Input Structures}
\label{sec:optimisationofinputstructures}

\section{Analysing Results}
\label{sec:analysingresults}

\section{Matching Structures}
\label{sec:matchingstructures}

\section{Apps}
\label{sec:apps}
