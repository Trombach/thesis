%methods

\part{Methods}
\label{sec:methods}

\chapter{Quantum Chemical Programs}
\label{sec:quantumchemicalprograms}

\chapter{The Program \textsc{Spheres}}
\label{sec:theprogramspheres}

Chapters~\ref{sec:fromstickyhardspheretoLJtypeclusters} and
\ref{sec:thegregorynewtonclusters} required methods to optimise and analyse
arbitrary cluster structures based on a potential depending only on the
distance between two spheres. For this purpose the program package
\textsc{Spheres} was developed. The following chapters contain information
about the most important classes and functions of the program.

\section{General Structure}
\label{sec:generalstructure}

Most of the functions and classes of the program are contained in header files,
that can be included in any code file. This allows the package to be quite
flexible while maintaining readability of the code.

The most basic header file is \texttt{geometry.h} which contains the classes
responsible for holding triplets of coordinates (\texttt{coord3d}) and to
perform basic 3x3 matrix algebra (\texttt{matrix3d}).

\section{The \texttt{structure} Class}
\label{sec:thestructureclass}



\section{The \texttt{pairPotential} Class}
\label{sec:thepairpotentialclass}

\section{Optimisation of Input Structures}
\label{sec:optimisationofinputstructures}

\section{Analysing Results}
\label{sec:analysingresults}

\section{Matching Structures}
\label{sec:matchingstructures}

\section{Apps}
\label{sec:apps}
